%!Tex Root = ./main.tex
\documentclass[french]{book}

\usepackage[T1]{fontenc}
\usepackage[utf8]{inputenc}
\usepackage{lmodern}
\usepackage[a4paper,margin=2cm]{geometry}
\usepackage{babel}

\usepackage{amsmath, amsfonts, amssymb, mathrsfs, mathtools}
\usepackage{esvect}
\usepackage{systeme}
\usepackage{braket}

\usepackage{fontspec}

% \usepackage{fouriernc}

\usepackage{helvet}
\usepackage{xcolor}
\usepackage{titlesec}
\usepackage{titling}
\usepackage{graphicx}
\usepackage{hyperref}

\usepackage{caption}  
\usepackage{subcaption}

\usepackage{import}
\usepackage{xifthen}
\usepackage{pdfpages}
\usepackage{transparent}

\newcommand{\incfig}[1]{%
    \def\svgwidth{\columnwidth}
    \import{./figures/}{#1.pdf_tex}
}


\hypersetup{
	colorlinks=true,
	linkcolor=blue,
	filecolor=magenta,      
	urlcolor=cyan,
	pdfpagemode=FullScreen,
}
\setlength{\fboxrule}{1pt}

\usepackage{tikz, tikz-cd, pgfplots}
\usetikzlibrary{positioning}
\usetikzlibrary{calc}

\newfontfamily\headingfont[]{Helvetica Bold}
% \setmainfont{Helvetica Light}

\titleformat{\chapter}
	{\headingfont\Huge\bfseries}
	{\thechapter}{1em}{\Huge}

\titleformat{\section}
	{\headingfont\LARGE\bfseries}
	{\thesection}{1em}{}
	
\titleformat{\subsection}
	{\headingfont\large\bfseries}
	{\thesubsection}{1em}{}
	

% \pretitle{\begin{center}\LARGE\bfseries}
	% \posttitle{\par\end{center}}
% \preauthor{\begin{center}\large}
	% \postauthor{\par\end{center}}
% \predate{\begin{center}\large}
	% \postdate{\par\end{center}}
	
\newcommand{\R}{\ensuremath{\mathbb{R}}}
\newcommand{\Z}{\ensuremath{\mathbb{Z}}}
\newcommand{\N}{\ensuremath{\mathbb{N}}}
\newcommand{\K}{\ensuremath{\mathbb{K}}}
\newcommand{\ra}{\ensuremath{\rightarrow}}
\newcommand{\lra}{\ensuremath{\longrightarrow}}


\renewcommand{\b}[1]{\ensuremath{\mathbf{#1}}}

\newcommand{\ps}[2]{\ensuremath{\prescript{}{#1}{#2}}}


\DeclareMathOperator{\rang}{rang\hspace{0.8pt}}
\newcommand{\id}[1][]{\ensuremath{\mathrm{Id_{#1}}}}
\DeclareMathOperator{\aut}{Aut}
\DeclareMathOperator{\supp}{supp}
\DeclareMathOperator{\sym}{Sym}
\DeclareMathOperator{\dist}{dist}
\DeclareMathOperator{\im}{Im}
\DeclareMathOperator{\ev}{ev}
\DeclareMathOperator{\sgn}{sgn}
% \DeclareMathOperator{\L}{\mathcal{L}}

\renewcommand{\hom}{\ensuremath{\mathrm{Hom}}}
\renewcommand{\L}{\ensuremath{\mathcal{L}}}
% \renewcommand{\phi}{\varphi}
% \let\phi\varphi

\newcommand{\verteq}{\rotatebox{90}{$\,=$}}
\newcommand{\equalto}[2]{\underset{\scriptstyle\overset{\mkern4mu\verteq}{#2}}{#1}}


\newcommand{\class}[1]{\ensuremath{\mathscr{C}^{#1}}}

\DeclareMathOperator{\lenop}{\mathbf{L}}
\newcommand{\len}[2]{\ensuremath{\lenop_{#1}^{#2}}}

\DeclarePairedDelimiter{\abs}{\lvert}{\rvert}%
\DeclarePairedDelimiter{\norm}{\lVert}{\rVert}%

\newcommand{\deriv}[1]{\ensuremath{\frac{\mathrm{d}}{\mathrm{d}#1}}}
\newcommand{\defeq}{\stackrel{\text{def}}{=}}


\title{\headingfont Théorie des groupes et algèbre multilinéaire}	
\author{Karol Gromada}
\date{2024-2025}

\usepackage{xcolor, tcolorbox}
\tcbuselibrary{most}
\usepackage{amsthm, amsmath}
\usepackage{xparse}
\usepackage{etoolbox} 

% \usepackage{cmbright}
\usepackage{mathpazo}
\usepackage[T1]{fontenc}

% Define colors
\definecolor{defcolor}{RGB}{173,0,230} % light blue
\definecolor{propcolor}{HTML}{733B1F}
\definecolor{thmcolor}{HTML}{733B1F}
\definecolor{corcolor}{HTML}{733B1F} % pink
% \definecolor{lemcolor}{RGB}{173, 114, 40} % pink
\definecolor{lemcolor}{HTML}{733B1F}
\definecolor{lembordercolor}{HTML}{94330F}
\definecolor{notcolor}{RGB}{84, 0, 209}
\definecolor{remcolor}{RGB}{191, 25, 69}


\newcounter{ctr}[subsection]
\renewcommand{\thectr}{\thesubsection.\arabic{ctr}}

\renewcommand\textsc[1]{\uppercase{{\scriptsize #1}}}

\NewDocumentEnvironment{definition}{o} % No arguments
{
	\refstepcounter{ctr}
  \begin{tcolorbox}[
    enhanced,                      % Enable advanced features
    colback=defcolor!4,           % Background color at 50% opacity
    colframe=white,                % Frame color (set to white or as desired)
    boxrule=0pt,                   % No default frame border
    left=7pt,                      % No left padding
    right=4.5pt,                     % No right padding
    top=2pt,                       % Top padding
    bottom=2pt,                    % Bottom padding
    overlay={                      % Add custom graphics on top
      \fill[defcolor] (frame.south west) 
        rectangle ([xshift=2.5pt]frame.north west); % Left colored edge
    },
    sharp corners,                 % Sharp (non-rounded) corners
    breakable,                     % Allow box to break across pages
    before skip=10pt,              % Space before the box
    after skip=10pt,               % Space after the box
    boxsep=5pt                     % Separation between box and content
  ]
  \bfseries{\textcolor{defcolor}{Définition \thectr\IfValueT{#1}{ \normalfont(#1)}.}}\normalfont % Prefix with "Definition [number]"
}{
  \end{tcolorbox}
}

\NewDocumentEnvironment{cor}{o} % No arguments
{
	\refstepcounter{ctr}
  \begin{tcolorbox}[
    enhanced,                      % Enable advanced features
		skin=bicolor,
    colback=lembordercolor!8,           % Background color at 50% opacity
		colbacklower=lemcolor!2,
    colframe=white,                % Frame color (set to white or as desired)
    boxrule=0pt,                   % No default frame border
    left=7pt,                      % No left padding
    right=0pt,                     % No right padding
    top=2pt,                       % Top padding
    bottom=2pt,                    % Bottom padding
    overlay={                      % Add custom graphics on top
      \fill[lembordercolor] (frame.south west)
        rectangle ([xshift=2.5pt]frame.north west); % Left colored edge
    },
    sharp corners,                 % Sharp (non-rounded) corners
    breakable,                     % Allow box to break across pages
    before skip=10pt,              % Space before the box
    after skip=10pt,               % Space after the box
    boxsep=5pt                     % Separation between box and content
  ]
  \bfseries{\textcolor{lemcolor}{Corollaire \thectr\IfValueT{#1}{ \normalfont(#1)}.}}\normalfont % Prefix with "Definition [number]"
}{
  \end{tcolorbox}
}


\NewDocumentEnvironment{lemme}{o} % No arguments
{
	\refstepcounter{ctr}
  \begin{tcolorbox}[
    enhanced,                      % Enable advanced features
		skin=bicolor,
    colback=lembordercolor!8,           % Background color at 50% opacity
		colbacklower=lemcolor!2,
    colframe=white,                % Frame color (set to white or as desired)
    boxrule=0pt,                   % No default frame border
    left=7pt,                      % No left padding
    right=0pt,                     % No right padding
    top=2pt,                       % Top padding
    bottom=2pt,                    % Bottom padding
    overlay={                      % Add custom graphics on top
      \fill[lembordercolor] (frame.south west)
        rectangle ([xshift=2.5pt]frame.north west); % Left colored edge
    },
    sharp corners,                 % Sharp (non-rounded) corners
    breakable,                     % Allow box to break across pages
    before skip=10pt,              % Space before the box
    after skip=10pt,               % Space after the box
    boxsep=5pt                     % Separation between box and content
  ]
  \bfseries{\textcolor{lemcolor}{Lemme \thectr\IfValueT{#1}{ \normalfont(#1)}.}}\normalfont % Prefix with "Definition [number]"
}{
  \end{tcolorbox}
}

\NewDocumentEnvironment{propo}{o} % No arguments
{
	\refstepcounter{ctr}
  \begin{tcolorbox}[
    enhanced,                      % Enable advanced features
		skin=bicolor,
    colback=lembordercolor!8,           % Background color at 50% opacity
		colbacklower=lemcolor!2,
    colframe=white,                % Frame color (set to white or as desired)
    boxrule=0pt,                   % No default frame border
    left=7pt,                      % No left padding
    right=0pt,                     % No right padding
    top=2pt,                       % Top padding
    bottom=2pt,                    % Bottom padding
    overlay={                      % Add custom graphics on top
      \fill[lembordercolor] (frame.south west)
        rectangle ([xshift=2.5pt]frame.north west); % Left colored edge
    },
    sharp corners,                 % Sharp (non-rounded) corners
    breakable,                     % Allow box to break across pages
    before skip=10pt,              % Space before the box
    after skip=10pt,               % Space after the box
    boxsep=5pt                     % Separation between box and content
  ]
  \bfseries{\textcolor{lemcolor}{Proposition \thectr\IfValueT{#1}{ \normalfont(#1)}.}}\normalfont % Prefix with "Definition [number]"
}{
  \end{tcolorbox}
}


\NewDocumentEnvironment{theoreme}{o} % No arguments
{
	\refstepcounter{ctr}
  \begin{tcolorbox}[
    enhanced,                      % Enable advanced features
		skin=bicolor,
    colback=lembordercolor!8,           % Background color at 50% opacity
		colbacklower=lemcolor!2,
    colframe=white,                % Frame color (set to white or as desired)
    boxrule=0pt,                   % No default frame border
    left=7pt,                      % No left padding
    right=4.5pt,                     % No right padding
    top=2pt,                       % Top padding
    bottom=2pt,                    % Bottom padding
    overlay={                      % Add custom graphics on top
      \fill[lembordercolor] (frame.south west) 
        rectangle ([xshift=2.5pt]frame.north west); % Left colored edge
    },
    sharp corners,                 % Sharp (non-rounded) corners
    breakable,                     % Allow box to break across pages
    before skip=10pt,              % Space before the box
    after skip=10pt,               % Space after the box
    boxsep=5pt,                     % Separation between box and content
    % middle=3pt
  ]
  \bfseries{\textcolor{thmcolor}{Théorème \thectr\IfValueT{#1}{ \normalfont(#1)}.}}\normalfont % Prefix with "Definition [number]"
}{
  \end{tcolorbox}
}

\NewDocumentEnvironment{remarque}{o} % No arguments
{
  \begin{tcolorbox}[
    enhanced,                      % Enable advanced features
    colback=remcolor!0,           % Background color at 50% opacity
    colframe=white,                % Frame color (set to white or as desired)
    boxrule=0pt,                   % No default frame border
    left=7pt,                      % No left padding
    right=0pt,                     % No right padding
    top=2pt,                       % Top padding
    bottom=2pt,                    % Bottom padding
    overlay={                      % Add custom graphics on top
      \fill[remcolor] (frame.south west) 
        rectangle ([xshift=2.5pt]frame.north west); % Left colored edge
    },
    sharp corners,                 % Sharp (non-rounded) corners
    breakable,                     % Allow box to break across pages
    before skip=10pt,              % Space before the box
    after skip=10pt,               % Space after the box
    boxsep=5pt                     % Separation between box and content
  ]
  \bfseries{\textcolor{remcolor}{Remarque\IfValueT{#1}{ \normalfont(#1)}.}}\normalfont % Prefix with "Definition [number]"
}{
  \end{tcolorbox}
}



%\newenvironment{corollary}[1][]{%
%	\begin{trivlist}
%		\item[\hskip \labelsep {\textcolor{corcolor}{\textit{Corollary.}}}] #1
%	}{%
%	\end{trivlist}
%}

\NewDocumentEnvironment{notation}{ o }{%
	\refstepcounter{ctr}
	\begin{trivlist}
		\item[\hskip \labelsep {\textcolor{notcolor}{ \headingfont{ Notation ~\thectr{} \normalfont \IfNoValueTF{#1} {} {~(#1)} \headingfont.}}}]
	}{%
	\end{trivlist}
}

\NewDocumentEnvironment{exemple}{ o }{%
	\refstepcounter{ctr}
	\begin{trivlist}
		\item[\hskip \labelsep {{ \headingfont{ Exemple ~\thectr{} \normalfont \IfNoValueTF{#1} {} {~(#1)} \headingfont.}}}]
	}{%
	\end{trivlist}
}



%\newenvironment{notation}[1][]{%
%	\begin{trivlist}
%		\item[\hskip \labelsep {\textcolor{notcolor}{\textit{Notation.}}}] #1
%	}{%
%	\end{trivlist}
%}

\ExplSyntaxOff

% Optional proof environment
\newenvironment{preuve}{%
	\begin{trivlist}
		\item[\hskip \labelsep {{\textit{\bfseries\textsc{\scriptsize \headingfont Démonstration.}}}}]
	}{%
	\qed
	\end{trivlist}
}


\usepackage{enumerate}

\begin{document}


\begin{titlepage}
  % \newgeometry{top=3in, bottom=1in}
  \maketitle
  \thispagestyle{empty}
  \centering
  % \vfill
  LMAT1231	
  % \vfill
\end{titlepage}
\restoregeometry

\tableofcontents
\thispagestyle{empty}

\newpage


\part{Théorie des groupes}
\chapter{Groupes} 

\section{Une étude sociologique des groupes}
Commençons par définir la notion de groupe.
\begin{definition}
  Un groupe est la donnée d'un triplet $(G, *, e)$ avec $G$ un ensemble muni d'une fonction $\circ: G \times G \ra G: (g, h) \mapsto g * h$ appelée \textbf{loi de composition}, telle que :
  \begin{enumerate}
    \item la loi $*$ est associative, c'est-à-dire : $$\forall a, b, c \in G, a * (b * c) = (a * b) * c$$
    \item $G$ contient un élément $e$, appelé \textbf{élément neutre}, tel que $\forall g \in G$, on ait
      $$e * g = g = g * e$$ 
    \item Pour tout $g \in G$, il existe $g' \in G$ tel que  $g' * g = e$
  \end{enumerate}
\end{definition}

\begin{proposition}
  Soit $G$ un groupe.
  \begin{enumerate}[(i)]
    \item L'élément neutre est unique
    \item Pour tous $g, h \in G$, si $g * h = e$ alors $h * g = e$
    \item Pour tout $g \in G$, l'élément $g'$ issu de la condition 3 est unique
  \end{enumerate}

  \tcblower
  \begin{preuve}
    Ceci est une preuve valable...
  \end{preuve}
\end{proposition}

Le corollaire suivant est immédiat.
\begin{corol}
  Soit $G$ un groupe. Pour tout $g \in G$, il existe un unique élément $g' \in G$ tel que $g' * g = e = g * g'$.
\end{corol}
Cet élément est noté $g^{-1}$ et est appelé l'inverse de $g$.
\par Par exemple, le singleton $G = \left\{ e \right\}$, avec la loi $e * e = e$ est un groupe. C'est le groupe \textbf{trivial}.
\par Certains groupes ont la particularité que leur loi de composition est commutative. Définissons la notion de \textit{groupe commutatif}.
\begin{definition}
  Un groupe $G$ est dit \textbf{commutatif} (ou \textbf{abélien}) si pour tous $g, h \in G$, on a $g * h = h * g$.
\end{definition}
Il est important de garder en tête que les groupes commutatifs font figure
d’exception. On termine cette chapter avec une dernière observation élémentaire.

\begin{lemme}
  Soit $(G, *)$ un groupe. Pour tout élément $g \in G$ et tout entier $n > 0$, on a $(g^{-1})^n = (g^n)^{-1}$.
  \tcblower
  \begin{preuve}
    On procède par récurrence sur $n$. Le cas $n=1$ est trivial. Supposons que l'énonce est vrai pour $n$. Cela implique que $(g^{-1})^n * g^n = e$. On en déduit :
    \begin{align*}
      (g^{-1})^{n+1} * g^{n+1} &= \left( g^{-1} * (g^{-1})^n \right) * (g^n * g) \\
                               &= g^{-1} * \left( (g^{-1})^n * g^n \right) * g \\
                               &= g^{-1} * e * g \\
                               &= g^{-1} * g
                               &= e
    \end{align*}
    Donc $(g^{-1})^{n+1}$ est l'inverse de $g^{n+1}$.
  \end{preuve}
\end{lemme}

\section{Groupes de permutations}
Intéressons-nous maintenant au cas particulier de l'exemple du début de cours. Prenons $O$ un ensemble non vide, dépourvu de toute autre structure. Dans ce cas, le groupe $\aut(O)$ est constitué de toutes les permutations des points de $O$, c'est-à-dire de toutes les bijections de $O$ dans lui-même. On note ce groupe $\sym(O)$ et on l'appelle le \textbf{groupe symétrique} associe à $O$. 
\par Dans le cas ou $O = \left\{ 1, \dots, n \right\}$, on écrit plutôt $\sym(n)$ ou $S_n$.
\par Un élément $\sigma \in \sym(n)$ est une fonction bijective qui associe à tout entier $i \in \left\{ 1, \dots, n \right\}$ un entier $\sigma(i)$. Le neutre de $\sym(n)$ est bien sûr la permutation triviale $\id: i \mapsto i$. La loi de groupe est la composition des fonctions notée $\circ$. Dans nos notations, le symbole $\circ$ sera omis. On écrit donc $$\alpha \circ \beta = \alpha \beta$$ et on parle du \textit{produit} de $\alpha$ et $\beta$.
\par Il est pratique de noter la permutation $\sigma$ comme suit
$$ \sigma = \begin{pmatrix}
  1 & 2 & \cdots & n \\
  \sigma(1) & \sigma(2) & \cdots & \sigma(n)
\end{pmatrix} $$
\begin{definition}
  On dit que $x \in O$ est un point fixe de $\sigma$ si $\sigma(x) = x$.
\end{definition}
Par convention, nous pouvons omettre les points fixes dans la notation matricielle d'une permutation $\sigma \in \sym(n)$. Par exemple, pour $n=7$, la permutation
$$\sigma^{-1} = \begin{pmatrix}
1 & 3 & 4 & 7 \\
3 & 1 & 7 & 4
\end{pmatrix}$$
fixe 2, 5 et 6.
\begin{definition}
  Le \textbf{support} d'une permutation $\sigma$ est le complémentaire de l'ensemble de ses points fixes. On la note $\supp(\sigma)$. Deux permutations sont dites \textbf{disjointes} si leurs supports sont disjoints.
\end{definition}

\begin{proposition}
  Deux permutations disjointes commutent. Autrement dit, si $\alpha, \beta \in \sym(O)$ sont disjointes, alors $\alpha \beta = \beta \alpha$.

  \tcblower
  \begin{preuve}
    Soit $x \in O$ un point quelconque. Supposons d'abord que $x \in \supp(\alpha)$. Comme $\supp(\alpha)$ et $\supp(\beta)$ sont disjoints, $x \not \in \supp(\beta)$ ce qui signifie que $x$ est fixé par $\beta$. Donc $\alpha \beta(x) = \alpha(x)$.
    \par De plus, on sait que $\alpha(x) \in \supp(\alpha)$, sinon $\alpha(x)$ serait fixé par $\alpha$, ce qui impliquerait que $\alpha \alpha(x) = \alpha(x)$ et donc que $\alpha(x) = x$ et $x \not \in \supp(\alpha)$. De manière analogue, on peut appliquer l'argument précédent et déduire que $\beta \alpha(x) = \alpha(x)$. On obtient ainsi que $\alpha \beta(x) = \alpha(x) = \beta \alpha(x)$. En échangeant $\alpha$ et $\beta$, on obtient la même conclusion dans le cas ou $x \in \supp(\beta)$.
    \par Enfin, si $x \not \in \supp(\alpha)$ et $x \not \in \sup(\beta)$ alors $x$ est fixé à la fois par $\alpha$ et $\beta$ ce qui implique que $\alpha \beta(x) = \alpha(x) = x = \beta(x) = \beta \alpha(x)$.
    \par Et donc, $\alpha \beta = \beta \alpha$ quelque soit $x \in  O$.
  \end{preuve}
\end{proposition}
Définissons maintenant ce qu'est une permutation cyclique.
\begin{definition}
  Une permutation $\sigma \in \sym(O)$ est appelée un \textbf{cycle} (ou est dite \textbf{cyclique}) si pour tous $x, y \in \supp(\sigma)$, il existe $n \in \Z$ tel que $\sigma^n(x) = y$.
  \newline Lorsque $\supp(\sigma)$ est fini de cardinal $k$, on dit que $\sigma$ est un $k$-\textbf{cycle}.
\end{definition}

\begin{remarque}
  Si $\sigma$ est un $k$-cycle, et que $x \in \supp(\sigma)$, alors
  $$\sigma = \begin{pmatrix}
    x & \sigma(x) & \cdots & \sigma^{k-1}(x) \\
    \sigma(x) & \sigma^2(x) & \cdots & x
  \end{pmatrix}$$
  Vu que les cycles jouent un rôle important, il convient de leur réserver la notation
  $$\sigma = \begin{pmatrix} x & \sigma(x) & \sigma^2(x) & \cdots & \sigma^{k-1}(x) \end{pmatrix}$$
\end{remarque}

Notre prochain objectif est de montrer que dans le groupe $\sym(n)$, tout élément est un produit de cycles disjoints.
\begin{theoreme}
  Soit $n>0$ un entier. Pour toute permutation $\sigma \in \sym(n)$, il existe des cycles $\gamma_0, \dots, \gamma_l$ tels que:
  \begin{enumerate}[(i)]
    \item $\sigma = \gamma_0 \cdots \gamma_l$.
    \item pour $i \not = j$, les cycles $\gamma_i$ et $\gamma_j$ sont disjoints.
    \item pour tout $i$ on a $\supp(\gamma_i) \subseteq \supp(\sigma)$.
  \end{enumerate}
  \tcblower
  \begin{preuve}
    On pose $m = \abs{\supp(\sigma)}$ et on procède par récurrence sur ce $m$.
    \newline Si $m=0$, la permutation $\sigma$ est triviale. On peut la voir comme un 0-cycle. Elle est évidemment disjointe de toute autre permutation. L'énonce est donc vrai.
    \newline Supposons maintenant $m>0$, de sorte que $\supp(\sigma)$ est non vide. Soit $x \in \supp(\sigma)$ un point quelconque. Considérons la suite $$x, \sigma(x), \sigma^2(x), \dots$$ de points de $O = \left\{ 1, \dots, n \right\}$. Comme $O$ est fini, il doit exister des entiers $p > q \geq 0$ tels que $\sigma^p(x) = \sigma^q(x)$. En post-composant par $(\sigma^q)^{-1} = \sigma^{-q}$, on trouve que $\sigma^{p-q}(x) = x$. Soit maintenant $k \geq 0$ le plus petit entier positif ou nul tel que $\sigma^k(x)=x$. Les points de l'ensemble
    $$O(x) = \left\{ x, \sigma(x), \dots, \sigma^{k-1}(x) \right\}$$ 
sont deux à deux distincts et permutés cycliquement par $\sigma$. Définissons un $k$-cycle $\gamma_0$ en posant
$$\gamma_0 = \begin{pmatrix} x & \sigma(x) & \cdots & \sigma^{k-1}(x) \end{pmatrix}$$
Clairement, la permutation $\sigma$ préservé $O(x)$, c'est-à-dire
$$\sigma(y) \in O(x) \text{ pour tout point } y \in O(x)$$
et la restriction de $\sigma$ à $O(x)$ coïncide avec $\gamma_0$. Notons que
$$\supp(\gamma_0) = O(x) \subseteq \supp(\sigma).$$
En outre, la permutation $\gamma^{-1}_0 \sigma$ fixe chacun des points de $O(x)$. Comme $\gamma_0$ agit comme l'identité sur le complémentaire de $O(x)$, on en déduit que
$$\supp(\sigma) = O(x) \sqcup \supp(\gamma^{-1}_0 \sigma) $$
En particulier $\gamma_0$ est disjoint de la permutation $\gamma^{-1}_0 \sigma$. Dès lors, l'hypothèse de récurrence s'applique à $\gamma^{-1}_0 \sigma$, qui s'écrit donc comme un produit de cycles
$$\gamma^{-1}_0 \sigma = \gamma_1 \cdots \gamma_l$$
satisfaisant les conditions (ii) et (iii) du théorème. En multipliant l'égalité qui précède par $\gamma_0$ à gauche, on trouve $\sigma = \gamma_0 \gamma_1 \cdots \gamma_l$, ce qui prouve (i). Comme $\supp(\sigma)$ est la réunion disjointe de $O(x)$ et de $\supp(\gamma^{-1}_0 \sigma)$, les assertions (ii) et (iii) découlent de l'hypothèse de récurrence.
  \end{preuve}
\end{theoreme}

\begin{definition}
  Soit $\sigma \in \sym(n)$ et $\gamma_1, \dots, \gamma_l$ des cycles disjoints tels que $\sigma = \gamma_1 \cdots \gamma_l$. La représentation $\sigma = \gamma_1 \cdots \gamma_l$ est appelée \textbf{décomposition standard de $\sigma$ en un produit de cycles}. Cette écriture de $\sigma$ est unique à une permutation près de l'ordre des facteurs $\gamma_i$.
\end{definition}

\section{Transpositions et signature}

\part{Algèbre multilinéaire}
\let\phi\varphi

\chapter{Formes linéaires}
Soit $\mathbb{K}$ un corps, $V$ et $W$ des espaces vectoriels sur $\mathbb{K}$. On définit
$$\hom_\K(V, W) = \left\{ A : V \ra W \mid A \text{ linéaire} \right\}$$
Une toute première affirmation est que $\hom_\K(V, W)$ est un espace vectoriel sur $\K$. La preuve se fait très simplement. Soit $A, B \in \hom_\K(V, W)$. On définit l'addition comme
\begin{align*}
  A+B: V &\lra W \\
  v &\longmapsto (A+B)(v) \defeq A(v) + B(v)
\end{align*}
et la multiplication par un scalaire $\lambda \in \K$ comme
\begin{align*}
  \lambda A: V &\lra W \\
  v &\longmapsto (\lambda A)(v) \defeq \lambda \cdot (A(v))
\end{align*}
La preuve en découle trivialement.

\section{Formes linéaires}
\begin{definition}
  Une forme linéaire sur un espace vectoriel $V$ sur $\K$ est une application linéaire $$f: V \lra \K$$
\end{definition}
On définit aussi le dual a $V$ qui est $V^* = \left\{ \text{formes linéaires sur } V \right\} \hom_\K(V, \K)$
\begin{theoreme}\label{thm:dual}
  Soit $\dim(V) = n < \infty$ et $(e_1, \dots, e_n)$ une base ordonnée de $V$. Alors l'application
  \begin{align*}
    \phi: V^* &\lra \K^n \\
    f &\longmapsto \left(f(e_1), \dots, f(e_n)\right)
  \end{align*}
  est un isomorphisme linéaire donc $\dim(V^*) = n$.
  \par En outre, $V^*$ possède une seule base ordonnée $(e^*_1, \dots, e^*_n)$ définie par $$e^*_i(e_j) = \delta_{ij}$$ pour tout $i, j \in \left\{ 1, \dots, n \right\}$. C'est la base duale.
  \tcblower
  \begin{preuve}
    $\phi$ est linéaire, car $\forall f, g \in V^*$, $\forall \lambda, \mu \in \K$,
    \begin{align*}
      \phi(\lambda f + \mu g) &= \left( (\lambda f + \mu g)(e_1), \dots, \left( (\lambda f + \mu g)(e_n) \right) \right) \\
                              &= \left( \lambda f(e_1) + \mu g(e_1), \dots, \lambda f(e_n) + \mu g(e_n) \right) \\
                              &= \lambda \left( f(e_1), \dots, f(e_n) \right) + \mu \left( g(e_1), \dots, g(e_n) \right) \\
                              &= \lambda \phi(f) + \mu \phi(g)
    \end{align*}
    De plus, $\phi$ est injective car
    \begin{align*}
      \ker(\phi) &= \left\{ f \in V^* \mid (f(e_1), \dots, f(e_n)) = (0, \dots, 0) \right\} \\
                 &= \left\{ f \in V^* \mid f(v) = 0 \; \forall v \in V \right\} \\
                 &= \left\{  0\right\}
    \end{align*}
    Pour vérifier la surjectivité, définissons pour tout $i \in \left\{ 1, \dots, n \right\}$ une forme linéaire $e^*_i: V \ra \K$ telle que $e^*_i(e_j) = \delta_{ij}$. Observons que
    \begin{align*}
      \phi(e^*_i) &= (e^*_i(e_1), \dots, e^*_i(e_i), \dots, e^*_i(e_n)) \\
                  &= (0, \dots, 1, \dots, 0) \\
                  &= i\text{-eme vecteur de la base canonique de } \K^n
    \end{align*}
    Cela implique que $\K^n$ contient une base de $\K^n$ et contient dès lors tout $\K^n$ puisque $\phi$ est linéaire. On conclut donc que $\phi$ est un isomorphisme.
  \end{preuve}

\end{theoreme}

\begin{remarque}
  Soit $V, W$ deux espaces vectoriels sur $\K$. Fixons $f \in V^*$ et $w \in W$. Définissons ensuite
  \begin{align*}
    B_{(f, w)}: V &\lra W \\
    v&\longmapsto f(v)\cdot w
  \end{align*}
  Alors, $B_{(f,w)}$ est linéaire donc $B_{(f, w)} \in \hom_\K(V, W)$. Cette construction fourni une application
  \begin{align*}
    V^* \times W &\lra \hom_\K(V, W) \\
    (f,w)& \longmapsto B_{(f, w)}
  \end{align*}
  Cette application est bilinéaire. Elle dépend linéairement de chacune des 2 variables.
\end{remarque}

\section{Bidual}
Le bidual de $V$ est défini comme $V^{**} \defeq (V^*)^*$. Ses éléments sont parfois appelés des cocovecteurs.
\par Soit $v \in V$ fixe et
\begin{align*}
  \ev_v: V^* &\lra \K \\
  f&\longmapsto f(v)
\end{align*}
Cette application $\ev_v$ est un cocovecteur. En effet, $\forall f, g \in V^*$ et $\forall \lambda, \mu \in \K$ 
\begin{align*}
  \ev_v(\lambda f + \mu g) &= (\lambda f + \mu g)(v) \\
                           &= \lambda f(v) + \mu g(v) \\
                           &= \lambda \ev_v(f) + \mu \ev_v(g)
\end{align*}
\begin{remarque}
  Tout vecteur est un cocovecteur.
\end{remarque}

\begin{theoreme}[Bidual]
  \begin{enumerate}[(i)]
    \item L'application $V \ra V^{**} : v \mapsto \ev_v$ est linéaire et injective.
    \item Si $\dim(V) = n < \infty$ alors, elle est aussi surjective (et donc c'est un isomorphisme).
  \end{enumerate}

  \tcblower
  \begin{preuve}
    (i) Soient $v,w \in V$, $\lambda, \mu \in \K$ et $f \in V^*$. On a
    \begin{align*}
      \ev_{\lambda v + \mu g}(f) &= f(\lambda v + \mu w) \\
                                 &= \lambda f(v) + \mu f(w) \\
                                 &= \lambda\ev_v(f) + \mu\ev_w(f) \\
                                 &= (\lambda\ev_v + \mu\ev_w)(f)
    \end{align*}
    Donc $v\mapsto \ev_v$ est linéaire.
    \par Supposons $v \in V \setminus \{ 0 \}$. On peut alors choisir une base $Q$ de $V$ qui contient $v$. Il existe alors un seul unique $f \in V^*$ tel que $f(v) = 1$ et  $f(w) = 0$ pour tout  $w \in Q \setminus \left\{ v \right\}$. Donc
    $$\ev_v(f) = 1 \not = 0 \implies \ev_v \not = 0 \implies v \not \in \ker(\ev)$$
    Donc $\ker(\ev) = \left\{ 0 \right\}$ et $\ev$ est injective.
    \par (ii) Par le théorème \ref{thm:dual} on sait que $\dim(V^*) = n$. De manière analogue, $\dim(V^{**}) = \dim(V^*) = n$. Par le théorème du rang, elle est aussi surjective.
  \end{preuve}
\end{theoreme}

\section{Transposée}
Considérons maintenant deux espaces vectoriels $V,W $ sur  $\K$ et une application $A \in \hom_\K(V, W)$.
\begin{definition}
  L'application
  \begin{align*}
    A^\top : W^* &\lra V^* \\
    f&\longmapsto f \circ A
  \end{align*}
  est appelée la transposée de $A$.
\end{definition}

\begin{theoreme}
  Soit $A \in \hom_\K(V, W)$.
  \begin{enumerate}[(i)]
    \item La transposée $A^\top: W^* \ra V^*$ est linéaire.
    \item $\forall B \in \hom_\K(V, W)$ on a $(A+B)^\top = A^\top + B^\top$.
    \item $\forall B \in \hom_\K(U, V)$ on a $(A\circ B)^\top = B^\top \circ A^\top$. 
    \item Si $\dim(V), \dim(W) < \infty$ alors $A$ est bijective si et seulement si  $A^\top$ l'est et $(A^{-1})^\top = (A^\top)^{-1}$.
  \end{enumerate}

  \tcblower
  \begin{preuve}
    (i) Soient $\phi, \psi \in W^*$ des formes linéaires et $\lambda, \mu \in \K$ des scalaires. On a
    \begin{align*}
      A^\top(\lambda\phi + \mu\psi) &= (\lambda\phi + \mu\psi) \circ A \\
                                    &= \lambda\phi\circ A + \mu\psi\circ A \\
                                    &= \lambda A^\top(\phi) + \mu A^\top(\psi)
    \end{align*}
    ce qui confirme que $A^\top$ est linéaire.
    \par (ii) Soit $\phi \in W^*$. On a
    \begin{align*}
      (A+B)^\top(\phi) &= \phi \circ (A+B) \\
                       &= \phi\circ A + \phi\circ B \\
                       &= A^\top(\phi) + B^\top(\phi) \\
                       &= (A^\top + B^\top)(\phi)
    \end{align*}
    \par (iii) Soit $\phi \in W^*$. On a
    \begin{align*}
      (A\circ B)^\top(\phi) &= \phi \circ (A \circ B) \\
                            &= (\phi \circ A) \circ B \\
                            &= A^\top(\phi)\circ B \\
                            &= B^\top\left(A^\top(\phi)\right) \\
                            &= (B^\top \circ A^\top)(\phi)
    \end{align*}
    \par (iv) Remarquons d'abord que si $\id: V \ra V : v \mapsto v$ est l'application identique alors  $\id^\top$ est l'application identique sur  $V^*$. En effet, pour toute forme linéaire  $\phi \in V^*$, on a $\id^\top(\phi) = \phi \circ \id = \phi$. Si maintenant $A: V \ra W$ est inversible, elle admet un inverse $A^{-1}$ et on a $\id = A^{-1}\circ A$ ce qui implique que $\id^\top = A^\top \circ (A^{-1})^\top$ par le point (iii). Cela confirme que $A^\top$ est inversible et que son inverse est $(A^{-1})^\top$. La réciproque s'établit de manière analogue.
  \end{preuve}
\end{theoreme}

\let\cleardoublepage\clearpage
\chapter{Application multilinéaire et produit tensoriel}
\section{Application bilinéaire}

\begin{definition}
  Soit $V_1, V_2, W$ des espaces vectoriels sur $\K$. Une application $f: V_1 \times V_2 \ra W$ est bilinéaire si elle est linéaire en chacune de ses coordonnés. C'est-à-dire que $\forall v_2 \in V_2$ fixé, $V_1 \ra W : x \mapsto f(x, v_2)$ est linéaire et $\forall v_1\in V_1$ fixé, $V_2\ra W : y \mapsto f(v_1, y)$ est linéaire.
\end{definition}
Donnons quelques exemples d'applications bilinéaires.
\par \textbf{Exemple 1:} Soit $\K = \R$,  $V_1 = V_2 = \R^n$, $W = \R$ alors le produit scalaire
\begin{align*}
  (\cdot, \cdot): \R^n \times \R^n &\lra \R \\
  \left( \vv{x} = \begin{pmatrix}x_1 \\ \vdots \\ x_n\end{pmatrix}, \vv{y} = \begin{pmatrix}y_1 \\ \vdots \\ y_n\end{pmatrix} \right)&\longmapsto \vv{x} \cdot \vv{y} \defeq \sum_{i=1}^n x_iy_i
\end{align*}
est une application (et même forme !) bilinéaire. Il y a plus d'exemples, mais flm d'écrire.
\par Introduisons la notation suivante.
$$\L(V_1, V_2; W) = \left\{ \text{application bilinéaire de } V_1 \times V_2 \text{ dans } W \right\}$$
\par \textbf{Affirmation:} $\L(V_1, V_2; E)$ est un espace vectoriel sur $\K$. La preuve se fait tres simplement a partir des définitions suivantes. Soit $f, g \in \L(V_1, V_2, W)$, on pose
\begin{gather*}
  (f+g)(v_1, v_2) = f(v_1, v_2) + g(v_1, v_2) \\
  (\lambda f)(v_1, v_2) = \lambda f(v_1, v_2)
\end{gather*}
On se pose donc la question naturelle quelle est la dimension de $\L(V_1, V_2; W)$? Fixons $(e_1, \dots, e_n)$ une base de  $V_1$ et $(d_1, \dots, d_m)$ une base de $V_2$. Tout $\phi \in \L(V_1, V_2; W)$ est déterminé par $\phi(e_i, d_j) = w_{ij} \in  W$. Soit $v_1 \in V_1$ et $v_2\in V_2$. On peut réécrire ces 2 vecteurs comme
$$v_1 = \sum_{i=1}^n \lambda_i e_i \quad (\lambda_i \in \K) \; \text{ et } \; v_2 = \sum_{j=1}^m \mu_j d_j \quad (\mu_j \in \K)$$
On a
\begin{align*}
  \phi(v_1, v_2) &= \phi\left( \sum_{i} \lambda_i e_i, \sum_j \mu_j d_j \right) \\
                 &= \sum_{i, j} \lambda_i \mu_j \phi(e_i, d_j) \\
                 &= \sum_{i, j} \lambda_i \mu_j w_{ij}
\end{align*}
Pour l'inverse, choisissons $n\cdot m$ vecteurs de $W$, disons $x_{ij} \in W$ avec $i \in \left\{ 1, \dots, n \right\}$, $j \in \left\{ 1, \dots, m \right\}$ alors $\exists ! \psi \in \L(V_1, V_2; W)$ tel que $\psi(e_i, d_j) = x_{ij}$. En effet, on peut poser
$$\psi\left( \underbrace{\sum \lambda_i e_i}_{v_1}, \underbrace{\sum \mu_j d_j}_{v_2} \right) \defeq \sum_{i,j} \lambda_i \mu_j x_{ij}$$
On voit ainsi que les éléments de l'espace auquel on est intéressé sont déterminés par $n\cdot m$ vecteurs de $W$ qui lui-même est un espace dont là d'immersion est, disons $l$. On conclut donc que
$$\dim\left( \L(V_1, V_2; W)\right) = n\cdot m \cdot l$$

\section{Application bilinéaire universelle}
Soit $V_1, V_2, U$ des espaces vectoriels sur $\K$.
\begin{definition}
  Une application $\phi \in \L(V_1, V_2; U)$ est appelée universelle si pour tout espace vectoriel $W$ sur $\K$ et $\forall h \in \L(V_1, V_2; W)$, $\exists! \tilde{h}: U \ra W$ linéaire telle que $\forall (v_1, v_2) \in V_1 \times V_2$ on a $h(v_1, v_2) = \tilde{h}(\phi(v_1, v_2))$.


\end{definition}
\begin{figure}[h]
  \centering
  \begin{tikzcd}
    {V_1 \times V_2} && U \\
                     && {\quad\quad\exists! \tilde{h}} \\
                     && W
                     \arrow["\varphi", from=1-1, to=1-3]
                     \arrow["h"', from=1-1, to=3-3]
                     \arrow[dashed, from=1-3, to=3-3]
  \end{tikzcd}
\end{figure}

\begin{theoreme}
  Il existe un espace vectoriel $U$ sur $\K$ ou $\phi \in \L(V_1, V_2; U)$ est universelle.

  \tcblower
  \begin{preuve}
    On va voir qu'on peut poser
    $$U = \L(V_1^*, V_2^*; \K)$$
    On doit construire
    \begin{align*}
      \phi: V_1 \times V_2 &\lra U \\
      (v_1, v_2) &\longmapsto \phi(v_1, v_2) : V_1^* \times V_2^* \lra \K
    \end{align*}
    On pose
    \begin{align*}
      \phi(v_1, v_2): V_1^* \times V_2^* &\lra \K \\
      (f_1, f_2) &\longmapsto f_1(v_1) \cdot f_2(v_2)
    \end{align*}
    \textbf{(Affirmation 1)} $\forall (v_1,v_2) \in V_1 \times V_2$, on a $\phi(v_1, v_2) \in U$.
    \par Autrement dit, le scalaire $\phi(v_1, v_2)(f_1, f_2) \in \K$ dépend linéairement de $f_1$ et $f_2$. Fixons $f_2 \in V_2^*$. Soit $\lambda, \lambda' \in \K$ et $g, g' \in V_1^*$. Alors on a
    \begin{align*}
      \phi(v_1,v_2)(\lambda g + \lambda' g', f_2) &= (\lambda g + \lambda' g')(v_1)\cdot f_2(v_2) \\
                                                  &= (\lambda g(v_1) + \lambda' g'(v_1)) \cdot f_2(v_2) \\
                                                  &= \lambda g(v_1)f_2(v_2) + \lambda' g'(v_1)f_2(v_2) \\
                                                  &= \lambda \phi(v_1,v_2)(g, f_2) + \lambda' \phi(v_1, v_2)(g', f_2)
    \end{align*}
    L'argument pour la linéarité par rapport à $f_2$ est analogue.
    \par \textbf{(Affirmation 2)} L'application
    \begin{align*}
      \phi: V_1 \times V_2 &\lra U \\
      (v_1, v_2) &\longmapsto \phi(v_1, v_2)
    \end{align*}
    est bilinéaire.
    \par Fixons $v_2 \in V_2$. Soit $\lambda, \mu \in \K$ et $x, y \in V_1$ et $(f_1, f_2) \in V_1^* \times V_2^*$. On a
    \begin{align*}
      \phi(\lambda x + \mu y, v_2)(f_1, f_2) &= f_1(\lambda x + \mu y) f_2(v_2) \\
                                             &= (\lambda f_1(x) + \mu f_1(y)) f_2(v_2) \\
                                             &= \lambda f_1(x)f_2(v_2) + \mu f_1(y)f_2(v_2) \\
                                             &= \lambda \phi(x, v_2)(f_1, f_2) + \mu \phi(y, v_2)(f_1,f_2) \\
                                             &= \left( \lambda\phi(x,v_2) + \mu\phi(y, v_2) \right)(f_1,f_2)
    \end{align*}

    Il reste à vérifier que $\phi$ est universelle. Pour démontrer ceci, construisons d'abord une base de $U$. Choisissons  $(e_1, \dots, e_n)$ et $(d_1, \dots, d_m)$ des bases ordonnées de $V_1$ et $V_2$.
    \par \textbf{(Affirmation 3)} L'ensemble $\left\{ \phi(e_i, d_j) \mid i \in \{1, \dots, n \}, j\in \{ 1, \dots, m\} \right\} \subseteq U$ est une base de $U$.
    \par Cet ensemble est linéairement indépendant. En effet, soit $\lambda_{ij} \in \K$ un choix de $nm$ scalaires tels que $$\sum_{i,j} \lambda_{ij} \phi(e_i, d_j) = 0.$$
    L'égalité implique que $\forall (f_1,f_2) \in V_1^* \times V_2^*$ on a
    $$\sum_{i,j} \lambda_{ij} \phi(e_i, d_j)(f_1,f_2) = 0$$
    Prenons le cas particulier ou $f_1 = e^*_s$ et $f_2 = d^*_t$. Alors on a
    $$0 = \sum_{i,j} \lambda_{ij} \phi(e_i, d_j)(e^*_s, d^*_t) = \sum_{i,j} \lambda_{ij} \underbrace{e^*_s(e_i)}_{\delta_{si}}\underbrace{d^*_t(d_j)}_{\delta_{tj}} = \lambda_{st}$$
    Montrons que les $\phi(e_i, d_j)$ formes une famille génératrice. Il y a $nm$ vecteurs dans cet ensemble. On observe que
    \begin{align*}
      \dim(U) &= \dim(\L(V_1,V_2;\K)) \\
              &= \dim(V_1^*)\dim(V_2^*) \\
              &= \dim(V_1)\dim(V_2) \\
              &= nm
    \end{align*}
    On doit dès lors montrer encore que $\phi$ est universelle. Soit $W$ un espace vectoriel sur $\K$ et $h \in \L(V_1,V_2; W)$ donnés. On veut montrer que $\exists! \tilde{h}: U \ra W$ linéaire tel que
    \begin{equation}\label{eq:univ}
      h = \tilde{h} \circ \phi \tag{$*$}
    \end{equation}
    On sait que les $\phi(e_i, d_j)$ forment une base de $U$. De plus,
    \begin{equation}\label{eq:univ_2}
      \tilde{h}(\phi(e_i, d_j)) = h(e_i, d_j) \tag{$**$}
    \end{equation}
    est nécessaire pour que \eqref{eq:univ} soit vraie. L'existence et l'unicité sont donc vérifiées. Il reste a verifier que \eqref{eq:univ} est vraie pour tous les points du domaine.
    \par Soit $(v_1,v_2) \in V_1\times V_2$, $v_1=\sum_i \lambda_i e_i$ et $v_2=\sum_j \mu_j d_j$ avec $\lambda_i, \mu_j \in \K$. Alors, on a
    \begin{align*}
      \tilde{h} \circ \phi(v_1,v_2) &= \tilde{h}\left( \phi\left( \sum_i \lambda_i e_i, \sum_j \mu_j d_j \right) \right) \\
                                    &= \tilde{h}\left( \sum_{i,j} \lambda_i \mu_j \phi(e_i, d_j) \right) \\
                                    &= \sum_{i, j} \lambda_i \mu_j \tilde{h}(\phi(e_i, d_j)) \\
                                    &= \sum_{i, j} \lambda_i \mu_j h(e_i, d_j) \\
                                    &= h\left( \sum_i \lambda_i e_i, \sum_j \mu_j d_j \right) \\
                                    &= h(v_1,v_2)
    \end{align*}
    donc \eqref{eq:univ} est vérifiée.
  \end{preuve}
\end{theoreme}

Introduisons une notation.
\begin{notation}
  $U = V_1 \otimes V_2$ est le produit tensoriel de $V_1$ et $V_2$ et $\phi(v_1, v_2) = v_1 \otimes v_2$ est le produit tensoriel de $v_1$ et $v_2$.
\end{notation}
Le produit tensoriel des vecteurs $v_1$ et $v_2$ est donc l'application bilinéaire universelle
\begin{align*}
  \otimes: V_1 \times V_2 &\lra U \\
  (v_1,v_2)&\longmapsto v_1 \otimes v_2
\end{align*}
Le tenseur élémentaire $v_1\otimes v_2$ est aussi une forme bilinéaire sur $V_1^* \times V_2^*$
$$v_1\otimes v_2(f_1,f_2) = f_1(v_1)f_2(v_2)$$

\begin{theoreme}
  Le produit tensoriel
  \begin{align*}
    \otimes : V_1 \times V_2 &\lra U = V_1\otimes V_2 \\
    (v_1, v_2)&\longmapsto v_1\otimes v_2
  \end{align*}
  est univoquement déterminé (à isomorphisme près) par la propriété universelle.
\end{theoreme}

\begin{remarque}
  Attention. Une erreur courante est de penser que l'égalité
  $$V_1\otimes V_2 = \left\{ v_1 \otimes v_2 \mid v_1 \in V_1, v_2\in V_2 \right\}$$
  est vraie. Ce n'est \textbf{pas} le cas !
  \par Autrement dit, $\phi: V_1 \times V_2 \ra U$ n'est pas surjective. Par contre, $V_1\otimes V_2$ contient une base $\left\{ e_i\otimes d_j : i \in \{ 1, \dots, n\}, j \in \{ 1, \dots, m\} \right\}$
\end{remarque}

Soit
\begin{align*}
  -\otimes -: V_1^* \times V_2^* &\lra V_1^*\otimes V_2^* \\
  (f_1, f_2)&=\longmapsto f_1\otimes f_2
\end{align*}
l'application bilinéaire universelle sur $V_1^* \times V_2^*$. En outre, $V_1^* \otimes V_2^*$ s'identifie a $\L(V_1^{**}, V_2^{**}; \K) = \L(V_1, V_2l \K)$.
\par Soit $(f_1,f_2) \in V_1^* \times V_2^*$ et
\begin{align*}
  f_1\otimes f_2: V_1\times V_2 &\lra \K \\
  (v_1,v_2)&\longmapsto f_1(v_1)f_2(v_2)
\end{align*}
A titre d'exemple, disons que $V_1=V_2=\R^n$ et $(e_1, \dots, e_n)$ est la base canonique. Comment construire une forme bilinéaire sur $\R^n$? On vient de voir que le produit tensoriel de 2 formes linéaires est une forme bilinéaire.
\par Donc par exemple, prenons l'application
\begin{align*}
  e_3^* \otimes e_4^*: \R^n \times \R^n &\lra \R \\
  \left( \begin{pmatrix} x_1 \\ \vdots \\ x_n \end{pmatrix}, \begin{pmatrix} y_1 \\ \vdots \\ y_n \end{pmatrix}  \right)&\longmapsto x_3\cdot y_4
\end{align*}
Pour un autre exemple, le produit scalaire usuel sur $\R^n$ est $\vv{x}\cdot \vv{y} = \sum_{i=1}^n x_iy_i$. On peut l'ecrire comme
$$\begin{pmatrix} x_1 \\ \vdots \\ x_n \end{pmatrix} \cdot \begin{pmatrix} y_1 \\ \vdots \\ y_n \end{pmatrix} = \left( \sum_{i=1}^n e_i^* \otimes e_i^* \right)\left( \begin{pmatrix} x_1 \\ \vdots \\ x_n \end{pmatrix} , \begin{pmatrix} y_1 \\ \vdots \\ y_n \end{pmatrix}  \right) $$ 

\chapter{Application multilinéaire et algèbre tensorielle}
Commençons par définir ce qu'est une application $l$-linéaire.
\begin{definition}
  Soit $l\geq 1$ un entier et $V_1, \dots, V_l, W$ des espaces vectoriels sur $\K$.  Une application $l$-linéaire est une fonction $$f: V_1 \times \cdots \times V_l \lra W$$
  qu'est linéaire sur chacune des coordonnés. Si $W = \K$ on parle de forme $l$-linéaire.
\end{definition}

Soit $l, l', l'' \geq 1$ des entiers, $f\in \L(V_1, \dots, V_l; \K), g\in \L(V_1', \dots, V_{l'}'; \K)$ et $h\in \L(V_1'', \dots, V_{l''}''; \K)$ des formes multilinéaires. Alors
\begin{align*}
  f\otimes g: V_1\times \cdots V_l\times V_1'\times \cdots V_{l'}' &\lra \K \\
  (v_1, \dots, v_l, v_1', \dots, v_{l'}')&\longmapsto f(v_1,\dots,v_l)g(v_1', \dots, v_{l'}')
\end{align*}
est une forme $(l+l')$-linéaire. De plus, on a 
$$(f\otimes g)\otimes h = f \otimes (g \otimes h)$$
Ensuite, soit  $f_1, \dots, f_l$ des formes linéaires sur $V_1, \dots, V_l$ respectivement, c'est-à-dire $f_i \in V_i^*$. Alors $f_1 \otimes \cdots \otimes f_l \in \L(V_1, \dots, V_l; \K)$. En outre,
$$\dim(\L(V_1, \dots, V_l; \K)) = \dim(V_1)\cdot\dim(V_2)\cdots \dim(V_l)$$
Une base de cet espace est donnée par
$$\left\{ e_{1,i_1}^* \otimes \cdots \otimes e^*_{l, i_l}  \mid 1 \leq i_1 \leq \dim(V_1), \dots, 1 \leq i_l \leq \dim(V_l)\right\}$$
ou $\left\{ e^*_{j, i_j} \mid 1 \leq j \leq \dim(v_j) \right\}$ est une base duale à une base de $V_j$. Enfin, on a un isomorphisme linéaire $V_1^*\otimes \cdots \otimes V_l^* \cong \L(V_1, \dots, V_l; \K)$. Un cas particulier important de cela est quand on prend $V$ ou $V^*$ comme espace vectoriel considérés.

\begin{definition}
  Un tenseur d'espèce $p \choose q$ sur $V$ est une forme $(p+q)$-linéaire
  $$T: \underbrace{V^*\times\cdots\times V^*}_\text{$p$ fois} \times \underbrace{V\times\cdots\times V}_\text{$q$ fois} \lra \K$$
  On note  $T^p_q(V)$ l'espace des tenseurs d'espèce  $p \choose q$. On dit aussi que $T$ est $p$-fois covariant et $q$-fois contravariant.
\end{definition}
On donne quelques cas particuliers.
\par $T_0^1(V) = \hom_\K(V^*, \K) = V^{**} \cong V$ qui est l'espace des cocovecteurs.
\par  $T_1^0(V) = \hom_\K(V, \K) = V^*$ qui est l'espace des covecteurs.
\par $T_2^0(V) = \L(V, V; \K)$ qui est l'espace des formes bilinéaires sur  $V$.
\begin{align*}
  T^p_q(V) &= \L(\underbrace{V^*, \dots, V^*}_p, \underbrace{V, \dots, V}_q; \K) \\
           &\cong V\otimes \cdots\otimes V \otimes V^*\otimes \cdots \otimes V^*
\end{align*}
et aussi $\dim(T_q^p(V)) = (\dim(V))^{p+q}$
\par On obtient une base de  $T_q^p(V)$ en fixant une base  $(e_1, \dots, e_n)$ de $V$ et en considérant $\left\{ e_{i_1}\otimes \cdots \otimes e_{i_p}\otimes e_{j_i}^*\otimes \cdots \otimes e_{j_q}^* \right\}$

\chapter{Produit extérieur}
\section{Application symétrique, anti-symétrique et alternée}

\begin{definition}
  Soit $f: V^l \ra W$ une application $l$-linéaire. $f$ est \textbf{symétrique} si $\forall \sigma \in \sym(l)$, $\forall (v_1,\dots, v_l) \in V^l$ on a $$f(v_{\sigma(1)}, \dots, v_{\sigma(l)}) = f(v_1, \dots, v_l).$$
  $f$ est \textbf{anti-symétrique} si $\forall \sigma \in \sym(l)$, $\forall (v_1, \dots, v_l) \in V^l$ on a $$f(v_{\sigma(1)}, \dots, v_{\sigma(l)}) = \sgn(\sigma)f(v_1, \dots, v_l).$$
  $f$ est \textbf{alternée} si $\forall (v_1, \dots, v_l) \in V^l$, $\exists i, j \in \left\{ 1, \dots, l \right\}$ avec $i \not = j$ tel que si $v_i = v_j$ alors $f(v_1, \dots, v_l) = 0$.
\end{definition}

\begin{proposition}
  Toute application $l$-linéaire $f: V^l \ra W$ alternée est automatiquement anti-symétrique.
  \tcblower
  \begin{preuve}
    Commençons par prouver le cas $l=2$. Soit $x, y \in V$. Vu que par hypothèse $f$ est alternée, on a
    $$f(x-y, x-y) = 0$$
    $f$ est 2-linéaire donc on peut écrire
    $$0 = f(x, x) - f(y, x) - f(x, y) + f(y, y)$$
    qui donne l'égalité
    $$f(x, y) = -f(y, x)$$
    Traitons ensuite le cas $l \geq 2$. Soit  $(v_1, \dots, v_l) \in V^l$ et, soit $\sigma \in \sym(l)$. Si $\sigma$ est une transposition, disons $\sigma = (i \; j)$, on observe que
    \begin{align*}
      v_{\sigma(h)} &= v_h \\
      v_{\sigma(i)} &= v_j  & \forall h \in \{ 1, \dots, l\} \setminus \{i, j \} \\
      v_{\sigma(j)} &= v_i
    \end{align*}
    Donc tous les $v_h$ ou $h \not = i, j$ sont fixes. Par le cas $l=2$, on déduit que si $\sigma$ est une transposition alors $f(v_{\sigma(1)}, \dots, f_{\sigma(l)}) = -f(v_1, \dots, v_l)$.
    \par En général, $\sigma$ est un produit de transpositions, disons
    \begin{align*}
      \sigma &= \tau_1 \cdots \tau_n \\
      \gamma &= \tau_1 \cdots \tau_{n-1} \\
      \tau &= \tau_n = (i \; j)
    \end{align*}
    On a
    \begin{align*}
      f(v_{\sigma(1)}, \dots, v_{\sigma(l)}) &= f(v_{\gamma\tau(1)}, \dots, v_{\gamma\tau(l)}) \\
                                             &= f(v_{\gamma(1)}, \dots, v_{\gamma(j)}, \dots, v_{\gamma(i)}, \dots, v_{\gamma(l)}) \\
                                             &= -f(v_{\gamma(1)}, \dots, v_{\gamma(i)}, \dots, v_{\gamma(j)}, \dots, v_{\gamma(l)}) \\
                                             &= (-1)(-1)^{n-1}f(v_1, \dots, v_l) \\
                                             &= (-1)^n f(v_1, \dots, v_l) \\
                                             &= \sgn(\sigma)f(v_1, \dots, v_l)
    \end{align*}
  \end{preuve}
\end{proposition}

Il est évident que toute fonction $f: \R \ra \R$ est la somme d'une fonction paire et impaire. En effet,
 $$f(x) = \frac{1}{2}(f(x) + f(-x)) + \frac{1}{2}(f(x) - f(-x))$$
 De manière analogue, toute forme bilinéaire sur $\R^n$ est la somme d'une forme bilinéaire symétrique et anti-symétrique. En effet, soit
 \begin{align*}
   f: \R^n \times \R^n &\lra \R \\
   (x, y)&\longmapsto f(x, y)
\end{align*}
alors, on peut écrire
$$f(x, y) = \frac{1}{2}(f(x, y) + f(y, x)) + \frac{1}{2}(f(x, y) - f(y, x))$$

\section{Produit extérieur}
\begin{definition}
  Soit $f: V^l \ra W$ une application $l$-linéaire.
  \par La \textbf{symétrisée} de $f$ est
  $$S(f): (v_1, \dots, v_l) \mapsto \sum_{\sigma \in \sym(l)} f(v_{\sigma(1)}, \dots, v_{\sigma(l)})$$
  L'\textbf{anti-symétrisée} de $f$ est
  $$A(f): (v_1, \dots, v_l) \mapsto \sum_{\sigma \in \sym(l)} \sgn(\sigma) f(v_{\sigma(1)}, \dots, v_{\sigma(l)})$$
\end{definition}

\begin{theoreme}
  \begin{enumerate}
    \item $S(f): V^l \ra W$ est $l$-linéaire et symétrique.
    \item $A(f): V^l \ra W$ est $l$-linéaire et alternée (et donc aussi anti-symétrique).
  \end{enumerate}

  \tcblower
  \begin{preuve}
    $S(f)$ et $A(f)$ sont $l$-linéaire, car elles sont définies comme combinaisons linéaires de $l!$ formes $l$-linéaire. Tritons maintenant la seconde partie du théorème.
    \par (1) Soit $\gamma \in \sym(l)$ et $(v_1, \dots, v_l) \in V^l$. On a
    \begin{align*}
      S(f)(v_{\gamma(1)}, \dots, v_{\gamma(l)}) &= \sum_{\sigma \in \sym(l)} f(v_{\gamma\sigma(1)}, \dots, v_{\gamma\sigma(l)}) \\
                                                &= \sum_{\sigma' \in \sym(l)} f(v_{\sigma'(1)}, \dots, v_{\sigma'(l)}) \\
                                                &= S(f)(v_1, \dots, v_l)
    \end{align*}
    Donc le caractère symétrique de $S(f)$ est vérifié.
    \par Ensuite, soit $(v_1, \dots, v_l) \in V^l$. On doit vérifier que si $\exists i < j$ tel que $v_i = v_j$ alors $A(f)(v_1, \dots, v_l) = 0$. Posons 
    $$E = \left\{ \sigma \in \sym(l) \mid \sigma^{-1}(i) < \sigma^{-1}(j) \right\}$$
    et
    $$E' = \left\{ \sigma \in \sym(l) \mid \sigma^{-1}(i) > \sigma^{-1}(j) \right\}$$
    Par définition, $\sym(l) = E \sqcup E'$.
    \begin{align*}
      A(f)(v_1, \dots, v_l) &= \sum_{\sigma \in \sym(l)} \sgn(\sigma)f(v_{\sigma(1)}, \dots, v_{\sigma(l)}) \\
                            &= \underset{\defeq T}{\sum_{\sigma \in E} \sgn(\sigma)f(v_{\sigma(1)}, \dots, v_{\sigma(l)})} + \underset{\defeq T'}{\sum_{\sigma \in E'} \sgn(\sigma)f(v_{\sigma(1)}, \dots, v_{\sigma(l)})}
    \end{align*}
    Notre but est dès lors de montrer que $T' = -T$.
    \par Observons que $E$ et $E'$ sont en bijection. Posons $\tau = (i \; j)$. On a
     \begin{align*}
       \sigma \in E &\iff \sigma^{-1}(i) < \sigma^{-1}(j) \\
                    &\iff \sigma^{-1}\tau(j) < \sigma^{-1}\tau(i) \\
                    &\iff \sigma^{-1}\tau^{-1}(j) < \sigma^{-1}\tau^{-1}(i) \\
                    &\iff (\tau\sigma)^{-1}(j) < (\tau\sigma)^{-1}(i) \\
                    &\iff (\tau\sigma)^{-1}(i) > (\tau\sigma)^{-1}(j) \\
                    &\iff \tau\sigma \in E'
    \end{align*}
    L'application $\sigma \mapsto \tau\sigma$ établi donc une bijection de $E$ vers $E'$.
    \par Par notre hypothèse de début, $v_i = v_j$ donc  $\forall k \in \left\{ 1, \dots, l \right\}$, on a
    $$v_{\tau(k)} = \begin{cases}
      v_k & \text{ si } k \not = i, j \\
      v_j = v_i & \text{ si } k = i \\
      v_i = v_j & \text{ si } k=j
    \end{cases}$$
    On voit donc que $v_{\tau(k)} = v_k$ pour tout  $k \in \left\{ 1, \dots, l \right\}$. On a
    \begin{align*}
      T' &= \sum_{\sigma' \in E'} \sgn(\sigma')f(v_{\sigma'(1)}, \dots, v_{\sigma'(l)}) \\
         &= \sum_{\sigma \in E} \sgn(\tau\sigma)f(v_{\tau\sigma(1)}, \dots, v_{\tau\sigma(l)}) \\
         &= -\sum_{\sigma \in E} \sgn(\sigma)f(v_{\sigma(1)}, \dots, v_{\sigma(l)})  \\
         &= -T
    \end{align*}
  \end{preuve}
\end{theoreme}

\begin{definition}
  Le produit extérieur de $l$ formes linéaires $f_1, \dots, f_l \in V^*$ est défini par
  $$f_1 \wedge f_2 \wedge \cdots \wedge f_l = A(f_1\otimes f_2 \otimes \cdots \otimes f_l.$$
\end{definition}
Pour un exemple de ce concept, prenons $V = \R^2 =\R^{2\times 1}$. On prend
 $$e_1^* \wedge e_2^* \in \L(\R^2, \R^2; \R)$$
 Par définition, on a
 \begin{align*}
   e_1^*\wedge e_2^* \left( \begin{pmatrix} x_1 \\ x_2\end{pmatrix}, \begin{pmatrix} y_1 \\ y_2\end{pmatrix} \right) &= A(e_1^* \otimes e_2^*)\left( \begin{pmatrix} x_1 \\ x_2\end{pmatrix}, \begin{pmatrix} y_1 \\ y_2\end{pmatrix} \right) \\
                                                                                                                      &= e_1^* \otimes e_2^* \left( \begin{pmatrix} x_1 \\ x_2\end{pmatrix}, \begin{pmatrix} y_1 \\ y_2\end{pmatrix} \right) - e_1^* \otimes e_2^* \left( \begin{pmatrix} y_1 \\ y_2\end{pmatrix}, \begin{pmatrix} x_1 \\ x_2\end{pmatrix} \right) \\
                                                                                                                      &= x_1y_2 - y_1x_2 \\
                                                                                                                      &= \det \begin{pmatrix} x_1 & y_1 \\ x_2 & y_2 \end{pmatrix}
 \end{align*}




\end{document}
