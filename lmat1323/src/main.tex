\documentclass[french]{article}

\usepackage[T1]{fontenc}
\usepackage[utf8]{inputenc}
\usepackage{lmodern}
\usepackage[a4paper,margin=2cm]{geometry}
\usepackage{babel}

\usepackage{amsmath, amsfonts, amssymb, mathrsfs, mathtools}
\usepackage{esvect}
\usepackage{systeme}
\usepackage{braket}

\usepackage{fontspec}

% \usepackage{fouriernc}

\usepackage{helvet}
\usepackage{xcolor}
\usepackage{titlesec}
\usepackage{titling}
\usepackage{graphicx}
\usepackage{hyperref}

\usepackage{caption}  
\usepackage{subcaption}

\hypersetup{
	colorlinks=true,
	linkcolor=blue,
	filecolor=magenta,      
	urlcolor=cyan,
	pdfpagemode=FullScreen,
}
\setlength{\fboxrule}{1pt}

\usepackage{tikz, tikz-cd, pgfplots}
\usetikzlibrary{positioning}
\usetikzlibrary{calc}

\newfontfamily\headingfont[]{Helvetica Bold}
% \setmainfont{Helvetica Light}

\usepackage{import}
\usepackage{xifthen}
\usepackage{pdfpages}
\usepackage{transparent}

\newcommand{\incfig}[1]{%
    \def\svgwidth{0.8\columnwidth}
    \import{./figures}{#1.pdf_tex}
}

\titleformat{\chapter}
	{\headingfont\Huge\bfseries}
	{\thechapter}{1em}{\Huge}

\titleformat{\section}
	{\headingfont\LARGE\bfseries}
	{\thesection}{1em}{}
	
\titleformat{\subsection}
	{\headingfont\large\bfseries}
	{\thesubsection}{1em}{}
	

\pretitle{\begin{center}\LARGE\bfseries}
	\posttitle{\par\end{center}\vskip 0.5em}
\preauthor{\begin{center}\large}
	\postauthor{\par\end{center}}
\predate{\begin{center}\large}
	\postdate{\par\end{center}\vskip 0.5em}
	
\newcommand{\R}{\ensuremath{\mathbb{R}}}
\newcommand{\Z}{\ensuremath{\mathbb{Z}}}
\newcommand{\U}{\ensuremath{\mathcal{U}}}
\newcommand{\N}{\ensuremath{\mathbb{N}}}
% \newcommand{\H}{\ensuremath{\mathcal{H}}}
\newcommand{\ra}{\ensuremath{\rightarrow}}
\newcommand{\lra}{\ensuremath{\longrightarrow}}


\renewcommand{\b}[1]{\ensuremath{\mathbf{#1}}}
\renewcommand{\H}{\ensuremath{\mathcal{H}}}

\newcommand{\ps}[2]{\ensuremath{\prescript{}{#1}{#2}}}


\DeclareMathOperator{\rang}{rang\hspace{0.8pt}}
\DeclareMathOperator{\id}{Id}
\DeclareMathOperator{\dist}{dist}
\DeclareMathOperator{\im}{Im}

\newcommand{\class}[1]{\ensuremath{\mathscr{C}^{#1}}}

\DeclareMathOperator{\lenop}{\mathbf{L}}
\newcommand{\len}[2]{\ensuremath{\lenop_{#1}^{#2}}}

\DeclarePairedDelimiter{\abs}{\lvert}{\rvert}%
\DeclarePairedDelimiter{\norm}{\lVert}{\rVert}%

\newcommand{\deriv}[1]{\ensuremath{\frac{\mathrm{d}}{\mathrm{d}#1}}}
\newcommand{\defeq}{\stackrel{\text{def}}{=}}

\renewcommand{\leq}{\leqslant}
\renewcommand{\geq}{\geqslant}
% \renewcommand{\epsilon}{\varepsilon}
\newcommand{\Tau}{\mathcal{T}}

\newcommand{\Int}{\mathrm{Int}}
\newcommand{\Cl}{\mathrm{Cl}}


\title{\headingfont Topologie}	
\author{Karol Gromada}
\date{2024-2025}

\usepackage{xcolor, tcolorbox}
\tcbuselibrary{most}
\usepackage{amsthm, amsmath}
\usepackage{xparse}
\usepackage{etoolbox} 

\usepackage{cmbright}
\usepackage[T1]{fontenc}

% Define colors
\definecolor{defcolor}{RGB}{173,0,230} % light blue
\definecolor{propcolor}{RGB}{255,165,0} % orange
\definecolor{thmcolor}{RGB}{0,128,128} % green
\definecolor{corcolor}{RGB}{3, 177, 252} % pink
% \definecolor{lemcolor}{RGB}{115, 59, 31} % brown
\definecolor{lemcolor}{HTML}{733B1F}
\definecolor{lembordercolor}{HTML}{94330F}
\definecolor{notcolor}{RGB}{84, 0, 209}
\definecolor{remcolor}{RGB}{191, 25, 69}


\newcounter{ctr}[subsection]
\renewcommand{\thectr}{\thesection.\arabic{ctr}}

\renewcommand\textsc[1]{\uppercase{{\scriptsize #1}}}

\NewDocumentEnvironment{definition}{o} % No arguments
{
	\refstepcounter{ctr}
  \begin{tcolorbox}[
    enhanced,                      % Enable advanced features
    colback=defcolor!5,           % Background color at 50% opacity
    colframe=white,                % Frame color (set to white or as desired)
    boxrule=0pt,                   % No default frame border
    left=7pt,                      % No left padding
    right=4.5pt,                     % No right padding
    top=2pt,                       % Top padding
    bottom=2pt,                    % Bottom padding
    overlay={                      % Add custom graphics on top
      \fill[defcolor] (frame.south west) 
        rectangle ([xshift=2.5pt]frame.north west); % Left colored edge
    },
    sharp corners,                 % Sharp (non-rounded) corners
    breakable,                     % Allow box to break across pages
    before skip=10pt,              % Space before the box
    after skip=10pt,               % Space after the box
    boxsep=5pt                     % Separation between box and content
  ]
  \bfseries\sffamily{\textcolor{defcolor}{Définition \thectr\IfValueT{#1}{ \normalfont(#1)}.}}\normalfont % Prefix with "Definition [number]"
}{
  \end{tcolorbox}
}


\NewDocumentEnvironment{corol}{o} % No arguments
{
	\refstepcounter{ctr}
  \begin{tcolorbox}[
    enhanced,                      % Enable advanced features
		skin=bicolor,
    colback=lembordercolor!8,           % Background color at 50% opacity
		colbacklower=lembordercolor!2,
    colframe=white,                % Frame color (set to white or as desired)
    boxrule=0pt,                   % No default frame border
    left=7pt,                      % No left padding
    right=0pt,                     % No right padding
    top=2pt,                       % Top padding
    bottom=2pt,                    % Bottom padding
    overlay={                      % Add custom graphics on top
      \fill[lembordercolor] (frame.south west) 
        rectangle ([xshift=2.5pt]frame.north west); % Left colored edge
    },
    sharp corners,                 % Sharp (non-rounded) corners
    breakable,                     % Allow box to break across pages
    before skip=10pt,              % Space before the box
    after skip=10pt,               % Space after the box
    boxsep=5pt                     % Separation between box and content
  ]
  \bfseries\sffamily{\textcolor{lemcolor}{Corollaire \thectr\IfValueT{#1}{ \normalfont(#1)}.}}\normalfont % Prefix with "Definition [number]"
}{
  \end{tcolorbox}
}

\NewDocumentEnvironment{proposition}{o} % No arguments
{
	\refstepcounter{ctr}
  \begin{tcolorbox}[
    enhanced,                      % Enable advanced features
		skin=bicolor,
    colback=lembordercolor!8,           % Background color at 50% opacity
		colbacklower=lembordercolor!2,
    colframe=white,                % Frame color (set to white or as desired)
    boxrule=0pt,                   % No default frame border
    left=7pt,                      % No left padding
    right=0pt,                     % No right padding
    top=2pt,                       % Top padding
    bottom=2pt,                    % Bottom padding
    overlay={                      % Add custom graphics on top
      \fill[lembordercolor] (frame.south west) 
        rectangle ([xshift=2.5pt]frame.north west); % Left colored edge
    },
    sharp corners,                 % Sharp (non-rounded) corners
    breakable,                     % Allow box to break across pages
    before skip=10pt,              % Space before the box
    after skip=10pt,               % Space after the box
    boxsep=5pt                     % Separation between box and content
  ]
  \bfseries\sffamily{\textcolor{lemcolor}{Proposition \thectr\IfValueT{#1}{ \normalfont(#1)}.}}\normalfont % Prefix with "Definition [number]"
}{
  \end{tcolorbox}
}

\NewDocumentEnvironment{lemme}{o} % No arguments
{
	\refstepcounter{ctr}
  \begin{tcolorbox}[
    enhanced,                      % Enable advanced features
		skin=bicolor,
    colback=lembordercolor!8,           % Background color at 50% opacity
		colbacklower=lemcolor!2,
    colframe=white,                % Frame color (set to white or as desired)
    boxrule=0pt,                   % No default frame border
    left=7pt,                      % No left padding
    right=0pt,                     % No right padding
    top=2pt,                       % Top padding
    bottom=2pt,                    % Bottom padding
    overlay={                      % Add custom graphics on top
      \fill[lembordercolor] (frame.south west) 
        rectangle ([xshift=2.5pt]frame.north west); % Left colored edge
    },
    sharp corners,                 % Sharp (non-rounded) corners
    breakable,                     % Allow box to break across pages
    before skip=10pt,              % Space before the box
    after skip=10pt,               % Space after the box
    boxsep=5pt                     % Separation between box and content
  ]
  \bfseries\sffamily{\textcolor{lemcolor}{Lemme \thectr\IfValueT{#1}{ \normalfont(#1)}.}}\normalfont % Prefix with "Definition [number]"
}{
  \end{tcolorbox}
}

\NewDocumentEnvironment{theoreme}{o} % No arguments
{
	\refstepcounter{ctr}
  \begin{tcolorbox}[
    enhanced,                      % Enable advanced features
		skin=bicolor,
    colback=lembordercolor!8,           % Background color at 50% opacity
		colbacklower=lembordercolor!2,
    colframe=white,                % Frame color (set to white or as desired)
    boxrule=0pt,                   % No default frame border
    left=7pt,                      % No left padding
    right=4.5pt,                     % No right padding
    top=2pt,                       % Top padding
    bottom=2pt,                    % Bottom padding
    overlay={                      % Add custom graphics on top
      \fill[lembordercolor] (frame.south west) 
        rectangle ([xshift=2.5pt]frame.north west); % Left colored edge
    },
    sharp corners,                 % Sharp (non-rounded) corners
    breakable,                     % Allow box to break across pages
    before skip=10pt,              % Space before the box
    after skip=10pt,               % Space after the box
    boxsep=5pt                     % Separation between box and content
  ]
  \bfseries\sffamily{\textcolor{lemcolor}{Théorème \thectr\IfValueT{#1}{ \normalfont(#1)}.}}\normalfont % Prefix with "Definition [number]"
}{
  \end{tcolorbox}
}

\NewDocumentEnvironment{remarque}{o} % No arguments
{
  \begin{tcolorbox}[
    enhanced,                      % Enable advanced features
    colback=remcolor!0,           % Background color at 50% opacity
    colframe=white,                % Frame color (set to white or as desired)
    boxrule=0pt,                   % No default frame border
    left=7pt,                      % No left padding
    right=0pt,                     % No right padding
    top=2pt,                       % Top padding
    bottom=2pt,                    % Bottom padding
    overlay={                      % Add custom graphics on top
      \fill[remcolor] (frame.south west) 
        rectangle ([xshift=2.5pt]frame.north west); % Left colored edge
    },
    sharp corners,                 % Sharp (non-rounded) corners
    breakable,                     % Allow box to break across pages
    before skip=10pt,              % Space before the box
    after skip=10pt,               % Space after the box
    boxsep=5pt                     % Separation between box and content
  ]
  \bfseries\sffamily{\textcolor{remcolor}{Remarque\IfValueT{#1}{ \normalfont(#1)}.}}\normalfont % Prefix with "Definition [number]"
}{
  \end{tcolorbox}
}

\NewDocumentEnvironment{notation}{o} % No arguments
{
  \begin{tcolorbox}[
    enhanced,                      % Enable advanced features
    colback=remcolor!0,           % Background color at 50% opacity
    colframe=white,                % Frame color (set to white or as desired)
    boxrule=0pt,                   % No default frame border
    left=7pt,                      % No left padding
    right=0pt,                     % No right padding
    top=2pt,                       % Top padding
    bottom=2pt,                    % Bottom padding
    overlay={                      % Add custom graphics on top
      \fill[remcolor] (frame.south west) 
        rectangle ([xshift=2.5pt]frame.north west); % Left colored edge
    },
    sharp corners,                 % Sharp (non-rounded) corners
    breakable,                     % Allow box to break across pages
    before skip=10pt,              % Space before the box
    after skip=10pt,               % Space after the box
    boxsep=5pt                     % Separation between box and content
  ]
  \bfseries\sffamily{\textcolor{remcolor}{Notation\IfValueT{#1}{ \normalfont(#1)}.}}\normalfont 
}{
  \end{tcolorbox}
}
% \NewDocumentEnvironment{proposition}{ o m }{%
% 	\refstepcounter{ctr}
% 	\begin{trivlist}
% 		\item[\hskip \labelsep {\textcolor{lemcolor}{ \headingfont{ Proposition ~\thectr{} \normalfont \IfValueT{#1}{~(#1)} \headingfont.}}}] \textit{#2}
% 	}{%
% 	\end{trivlist}
% }

\NewDocumentEnvironment{prop}{ o m }{%
	\refstepcounter{ctr}
	\begin{trivlist}
		\item[\hskip \labelsep {{ \headingfont{\textit{Propriété ~\thectr{}}\normalfont \IfValueT{#1}{~(#1)} \headingfont.}}}]\textit{#2}
	}{%
	\end{trivlist}
}

%\newenvironment{corollary}[1][]{%
%	\begin{trivlist}
%		\item[\hskip \labelsep {\textcolor{corcolor}{\textit{Corollary.}}}] #1
%	}{%
%	\end{trivlist}
%}


\NewDocumentEnvironment{exemple}{ o }{%
	\refstepcounter{ctr}
	\begin{trivlist}
		\item[\hskip \labelsep {{ \headingfont{ Exemple ~\thectr{} \normalfont \IfNoValueTF{#1} {} {~(#1)} \headingfont.}}}]
	}{%
	\end{trivlist}
}

\NewDocumentEnvironment{cor}{ o m }{%
	\refstepcounter{ctr}
	\begin{trivlist}
		\item[\hskip \labelsep {
			\IfValueTF{#1}{
				\begin{minipage}{\linewidth}
					\textcolor{corcolor}{\headingfont{Corollaire~\thectr{} \normalfont \IfValueT{#1}{~(#1)}\headingfont.}}
				\end{minipage}
			}{
				\textcolor{corcolor}{\headingfont{Corollaire~\thectr{} \normalfont \IfValueT{#1}{~(#1)}\headingfont.}}
			}}]
		\textit{#2}
	}{%
	\end{trivlist}
}


%\newenvironment{notation}[1][]{%
%	\begin{trivlist}
%		\item[\hskip \labelsep {\textcolor{notcolor}{\textit{Notation.}}}] #1
%	}{%
%	\end{trivlist}
%}

\ExplSyntaxOff

% Optional proof environment
\newenvironment{preuve}{%
	\begin{trivlist}
		\item[\hskip \labelsep {{\textit{\bfseries\textsc{\scriptsize \headingfont Démonstration.}}}}]
	}{%
	\qed
	\end{trivlist}
}


\begin{document}

\begin{titlepage}
  \newgeometry{top=3in, bottom=1in}
  \maketitle
  \thispagestyle{empty}
  \centering
  \vfill
  LMAT1323
  \vfill
\end{titlepage}
\restoregeometry

\tableofcontents
\thispagestyle{empty}

\newpage

\section{Espaces métriques}

\subsection{Espace métrique}
Définissons d'abord ce qu'est un espace métrique.
\begin{definition}
  Soit $X$ un ensemble et $d: X^2 \ra \R$ une fonction telle que
  \begin{itemize}
    \item[$D_1$] $d(x, y) = 0 \iff x = y$
    \item[$D_2$] $d(x, y) = d(y, x)$ pour tout $x, y \in X$
    \item[$D_3$] $d(x, y) + d(y, z) \geq d(x, z)$ pour tout $x, y, z \in X$  
  \end{itemize}
  alors, on dit que $d$ est une \textbf{\it métrique} sur $X$ et que $(X, d)$ est un \textbf{\it espace métrique}.
\end{definition}

\begin{lemme}
  Si $d: X^2 \ra \R$ est une métrique, alors $d(x, y) \geq 0$ pour tout $x, y \in X$.

  \tcblower
  \begin{preuve}
    Posons $z = x$ dans l'axiome $D_3$ pour obtenir 
    $$d(x, y) + d(y, x) \geq d(x, x) = 0$$ 
    Par $D_2$, $d(y, x) = d(x, y)$, donc on a $2d(x, y) \geq 0$. Vu que $2$ est inversible, on a $d(x, y) \geq 0$.
  \end{preuve}
\end{lemme}

Ensuite, donnons quelque exemples d'espaces métriques.
\begin{enumerate}
  \item Soit $X = \R^n$ et $$d(x, y) = \left( \sum_{i=1}^n (x_i - y_i)^2 \right)^\frac{1}{2}$$
  \item Soit $X = \R^n$ et $$d_1(x, y) = \sum_{i=1}^n |x_i - y_i| \quad\textrm{ (la métrique $l_1$)}$$
    $$d_p(x, y) = \left( \sum_{i=1}^n \abs{x_i - y_i}^p \right)^\frac{1}{p} \quad\textrm{ (la métrique $l_p$)}$$
  \item Soit $X = \R^n$ et $$d_\infty(x, y) = \max_{1 \leq i \leq n} \big\{ \abs{x_i - y_i} \big\} \quad\textrm{ (la métrique $l_\infty$)}$$
  \item Soit $X = V$ un espace vectoriel normé et $$d(u, v) = \norm{u - v}$$
  \item Soit $X = C([a, b])$ l'ensemble des fonctions continues sur $[a, b]$, $1 < p < \infty$ et $$d_p(f, g) = \left( \int_a^b \abs{f(t) - g(t)}^p dt\right)^\frac{1}{p} \quad\textrm{ (la métrique $l_p$ sur $C([a, b])$)} $$
    sinon,
    $$d_\infty(f, g) = \max_{a \leq t \leq b} \big\{ \abs{f(t) - g(t)} \big\} \quad\textrm{ (la métrique $l_\infty$ sur $C([a, b])$)}$$
  \item Soit un ensemble $X$ et 
    $$ d(x, y) = \begin{cases}
      0 & \textrm{si } x = y \\
      1 & \textrm{si } x \neq y
    \end{cases} \quad\textrm{ (métrique discrète sur $X$)}$$
\end{enumerate}

\begin{theoreme}
  Soient $(X_1, d_1), \dots, (X_n, d_n)$ des espaces métriques. Posons $$X = \prod_{i=1}^{n} X_i$$
  Pour chaque $n$-uplet de points $x = (x_1, \dots, x_n), y = (y_1, \dots, y_n) \in X$, on définit la métrique
  $d: X^2 \ra \R$ par
  $$ d(x, y) = \max_{1 \leq i \leq n} \big\{ d_i(x_i, y_i)\big\}$$
  alors $(X, d)$ est un espace métrique.

  \tcblower
  \begin{preuve}
    Vérifions les axiomes de la métrique.
    \begin{itemize}
      \item[$D_1$] Si $d(x, y) = 0$, alors $d_i(x_i, y_i) = 0$ pour tout $i$. Donc $x_i = y_i$ pour tout $i$, donc $x = y$.
        \newline Si $x = y$ alors $d_i(x_i, y_i) = 0$ pour tout $i$, donc $d(x, y) = 0$.
      \item[$D_2$] Puisque $d_i(x_i, y_i) = d_i(y_i, x_i)$ pour tout $i$, on a $d(x, y) = d(y, x)$.
      \item[$D_3$] Soit $z = (z_1, \dots, z_n) \in X$ et $j, k \in \N$ tels que
        $$ d(x, y) = d_j(x_j, y_j) $$
        $$ d(y, z) = d_k(x_k, y_k)$$
        Alors, pour $i \in \{1, \dots, n\}$, on a
        $$d_i(x_i, y_i) \leq d_j(x_j, y_j)$$
        $$d_i(y_i, z_i) \leq d_k(y_k, z_k)$$
        Vu que $d_i$ est une métrique, on a
        $$d_i(x_i, z_i) \leq d_i(x_i, y_i) + d_i(y_i, z_i) \leq d_j(x_j, y_j) + d_k(y_k, z_k) = d(x, y) + d(y, z)$$
        donc 
        $$d(x, z) = \max_{1 \leq i \leq n} \big\{ d_i(x_i, z_i) \big\} \leq d(x, y) + d(y, z)$$
    \end{itemize}
  \end{preuve}
\end{theoreme}

\begin{remarque}
  La métrique $d$ est appelée la \textbf{\it métrique produit} des métriques $d_1, \dots, d_n$.
\end{remarque}

Comparons maintenant les espaces métriques $\left(\R^2, d_\text{eucl} \right)$ et $\left(\R^2, d_\infty \right)$. Pour $a \in \R^2$ fixe, quels sont les ensembles
$$ \left\{ x \in \R^2 : d_\text{eucl}(x, a) \leq 1 \right\} \quad \text{et} \quad \left\{ x \in \R^2 : d_\infty(x, a) \leq 1 \right\} \text{ ?}$$

\begin{figure}[h]
  \centering
  % First Subfigure: Circle around point 'a'
  \begin{subfigure}[b]{0.45\textwidth}
      \centering
      \begin{tikzpicture}[scale=0.9, >=stealth]
          % Draw X and Y axes
          \draw[->, thick] (-1,0) -- (5,0) node[right] {$x$};
          \draw[->, thick] (0,-1) -- (0,5) node[above] {$y$};
          \node[below] at (2.5, -0.25) {$d_\text{eucl}(x, a) \leq 1$};

          % Define the coordinates of point 'a'
          \coordinate (a) at (3,2);

          % Draw point 'a'
          \fill (a) circle (2pt) node[above right] {$a$};

          % Draw a circle around point 'a' with radius 1
          \draw (a) circle [radius=1];
      \end{tikzpicture}
      \label{fig:circle}
  \end{subfigure}
  \hfill
  % Second Subfigure: Square around point 'a'
  \begin{subfigure}[b]{0.45\textwidth}
      \centering
      \begin{tikzpicture}[scale=0.9, >=stealth]
          % Draw X and Y axes
          \draw[->, thick] (-1,0) -- (5,0) node[right] {$x$};
          \draw[->, thick] (0,-1) -- (0,5) node[above] {$y$};
          \node[below] at (2.5, -0.25) {$d_\infty(x, a) \leq 1$};

          % Define the coordinates of point 'a'
          \coordinate (a) at (3,2);

          % Draw point 'a'
          \fill (a) circle (2pt) node[above right] {$a$};

          % Draw a square around point 'a' with side length 2 (centered at 'a')
          \draw ($(a) + (-1,-1)$) rectangle ($(a) + (1,1)$);
      \end{tikzpicture}
      \label{fig:square}
  \end{subfigure}
  \label{fig:combined}
\end{figure}

\begin{definition}\label{def:isom}
  Deux espaces métriques $(A, d_A)$ et $(B, d_B)$ sont \textbf{\it isométriques} s'il existe des fonctions réciproques $f: A \ra B$ et $g: B \ra A$ telles que pour chaque $x, y \in A$,
  $$ d_B(f(x), f(y)) = d_A(x, y)$$
  et pour chaque $u, v \in B$,
  $$ d_A(g(u), g(v)) = d_B(u, v)$$
\end{definition}

\begin{theoreme}
  Deux espaces métriques $(A, d_A)$ et $(B, d_B)$ sont isométriques si et seulement s'il existe $f: A \ra B$ telle que
  \begin{enumerate}
    \item $f$ est une bijection
    \item pour tout $x, y \in A$, $d_B(f(x), f(y)) = d_A(x, y)$
  \end{enumerate}

  \tcblower
  \begin{preuve}
    Prouvons tout d'abord l'implication vers la droite. Supposons que $(A, d_A)$ et $(B, d_B)$ sont isométriques. Il existe donc des fonctions réciproques $f: A \ra B$ et $g: B \ra A$ ce qui implique que $f$ est une bijection. De plus, $f$ satisfait la condition $2$ par \eqref{def:isom}.
    \newline Prouvons maintenant l'implication vers la gauche. Supposons que $f: A \ra B$ est une bijection telle que pour tout $x, y \in A$, $d_B(f(x), f(y)) = d_A(x, y)$. Vu que $f$ est bijection, elle est inversible. Soit $g: B \ra A$ telle que $f$ et $g$, déterminée en posant $$g(b) = a \text{ si  } f(a) = b$$ Pour tout $u, v \in B$, soient $x = g(u)$ et $y = g(v)$. Alors $$d_A(g(u), g(v)) = d_A(x, y)$$
    Par 2, on a $$d_A(x, y) = d_B(f(x), f(y)) = d_B(u, v)$$
  \end{preuve}
\end{theoreme}

\subsection{Fonctions continues sur les espaces métriques}

Défnissons tout d'abord ce qu'est une fonction continue grâce à une ancienne définition du cours LMAT1121.
\begin{definition}[Ancienne]
  Une fonction $f: \R \ra \R$ est \textbf{\it continue}, étant donné $y \in \R$ et $\varepsilon > 0$, nous pouvons trouver un $\delta(y, \varepsilon) > 0$ tel que pour tout $x \in \R$
  $$\abs{f(y) - f(x)} < \varepsilon \text{ si } \abs{y - x} < \delta(y, \varepsilon)$$
\end{definition}
Mais cette définition est trop rigide. Elle nous contraint à travailler dans l'espace $\R$ muni de la métrique euclidienne. Nous allons donc généraliser cette définition.

\begin{definition}[Nouvelle]
  Soit $(X, d)$ et $(Y, \rho)$ des espaces métriques. Une fonction $f: X \ra Y$ est \textbf{\it continue} si, étant donné $p \in X$ et $\varepsilon > 0$, il existe $\delta(p, \varepsilon) > 0$ tel que pour tout $x \in X$
  $$\rho(f(p), f(x)) < \varepsilon \text{ si } d(p, x) < \delta(p, \varepsilon)$$
\end{definition}
Voici quelques exemples...

\begin{lemme}
  Si $(X, d)$, $(Y, \rho)$ et $(Z, \sigma)$ sont des espaces métriques et $g: X \ra Y$ et $f: Y \ra Z$ sont des fonctions continues, alors $f \circ g: X \ra Z$ est continue.
\end{lemme}

\subsection{Ensembles ouverts dans les espaces métriques}
\begin{definition}
  Soit $(X, d)$ un espace métrique. Nous disons qu'un sous-ensemble $E$ est ouvert si, pour tout $e \in E$, nous pouvons trouver un $\delta > 0$ (qui dépend de $e$) tel que
  $$x \in E \text{ quand } d(x, e) < \delta$$
\end{definition}
Les ouverts de $(\R, d_\text{euc})$ sont des intervalles ouverts:
$$e \in \R : \{ x \in \R \mid d(x, e) < \delta \} = ]e -\delta, e + \delta[$$

\begin{definition}[Boule ouverte]
  Soit $(X, d)$ un espace métrique, $x \in X$ et $r > 0$. On définit la boule ouverte de centre $x$ et de rayon $r$ par
  $$B(x, r) = \{ y \in X \mid d(x, y) < r \}$$
\end{definition}
Prenons un exemple d'un ensemble qui n'est pas ouvert. Pour $(\R^n, d_\text{euc})$ et $x \in \R$, l'ensemble $\{ x \}$ n'est pas ouvert.
\newline Si $(X, d_\varepsilon)$ est un espace métrique avec la métrique discrète, alors
$$\{ x \} = B\left(x, \frac{1}{2}\right)$$
et tout les sous-ensembles de $X$ sont ouverts. Notons que $d(x, x) = 0 < \frac{1}{2}$ et $d(x, y) = 1 > \frac{1}{2}$ pour $x \neq y$. Si $x \in E \subset X$ alors $d(x, y) < \frac{1}{2}$ implique $y = x \in E$ et donc $E$ est ouvert.

\begin{theoreme}
  Soit $(X, d)$ un espace métrique. Alors
  \begin{enumerate}
    \item $\varnothing$ et $X$ sont ouverts.
    \item Si $U_\alpha$ est ouvert pour tout $\alpha \in A$, alors $\bigcup\limits_{\alpha \in A} U_\alpha$ est ouvert.
    \item Si $U_j$ est ouvert pour tout $1 \leq j \leq n$, alors $\bigcap\limits_{j=1}^{n} U_j$ est ouvert.
  \end{enumerate}

  \tcblower
  \begin{preuve}
    \begin{enumerate}
      \item Comme y a pas de points $e \in \varnothing$, l'affirmation
        $$x \in \varnothing \text{ quand } d(x, e) < \delta$$
        est vraie pour tout $e \in \varnothing$ donc $\varnothing$ est ouvert.
      \item Si $e \in \bigcup\limits_{\alpha \in A} U_\alpha$ alors on trouve $\alpha_1 \in A$ particulier tel que $e \in U_{\alpha_1}$. Comme $U_{\alpha_1}$ est ouvert, on peut trouver $\delta > 0$ tel que $x \in U_{\alpha_1}$ quand $d(x, e) < \delta$. 
        \newline Ensuite, comme $U_{\alpha_1} \subseteq \bigcup\limits_{\alpha \in A} U_\alpha$ on a $x \in \bigcup\limits_{\alpha \in A} U_\alpha$ quand $d(x, e) < \delta$ ce qui implique que $\bigcup\limits_{\alpha \in A} U_\alpha$ est ouvert.
      \item Si $e \in \bigcap\limits_{j=1}^{n} U_j$ alors $e \in U_j$ pour tout $1 \leq j \leq n$. Comme $U_j$ est ouvert, on peut trouver $\delta_j > 0$ tel que $x \in U_j$ quand $d(x, e) < \delta_j$.
        \newline Posons $\delta = \min\{\delta_1, \dots, \delta_n\}$. Clairement, $\delta > 0$ donc $x \in \bigcap\limits_{j=1}^{n} U_j$ quand $d(x, e) < \delta$ ce qui implique que $\bigcap\limits_{j=1}^{n} U_j$ est ouvert.
    \end{enumerate}
  \end{preuve}
\end{theoreme}
\begin{remarque}
  Une interesection infinie d'ouverts n'est pas nécessairement ouverte.
\end{remarque}
Considérons par exemple $\R$ muni de la métrique usuelle. Par exemple, pour tout $n \in \N_*$, l'ensemble $]-1, \frac{1}{n}[$ est ouvert mais l'intersection infinie $$\bigcap\limits_{n=1}^{\infty} ]-1, \frac{1}{n}[ \: = \: ]-1, 0]$$ n'est pas ouverte.
\newline Pour un autre exemple, considérons $(\R^n, d_\text{euc})$. Nous avons que $B\left(x, \frac{1}{j}\right)$ est ouvert mais l'intersection infinie
$$\bigcap_{j=1}^{\infty} B\left(x, \frac{1}{j}\right) = \{x\}$$
ne l'est pas.
\newline Pour éviter toute confusion, nous introduisons la notation suivante: soit $(X, d)$ et $(Y, \rho)$ des espaces métriques et $f: X \ra Y$ une fonction. Pour $U \subseteq Y$,
$$f^{-1}(U) = \{ x \in X \mid f(x) \in U \}$$
\begin{theoreme}
  Soit $(X, d)$ et $(Y, \rho)$ des espaces métriques. Une fonction $f: X \ra Y$ est continue si et seulement si $f^{-1}(U)$ est ouvert dans $X$ quand $U$ est ouvert dans $Y$.

  \tcblower
  \begin{preuve}
    Pour la première partie de la preuve, supposons que $f$ est continue et que $U \subseteq Y$ est ouvert. Si $x \in f^{-1}(U)$, on peut trouver $y \in U$ tel que $y=f(x)$.
    \newline Comme  $U \subseteq Y$ est ouvert, il existe $\varepsilon > 0$ tel que $z \in U$ quand $\rho(y, z) < \varepsilon$.
    \newline Comme $f$ est continue, il existe $\delta > 0$ tel que
    $$\rho(y, f(w)) = \rho(f(x), f(w)) < \epsilon \text{ quand } d(x, w) < \delta$$
    C'est a dire que $w \in f^{-1}(U)$ quand $d(x, w) < \delta$ ce qui implique que $f^{-1}(U)$ est ouvert.
    \newline Pour la deuxième partie de la preuve, supposons que $f^{-1}(U)$ est ouvert quand $U\subseteq Y$ est ouvert. Si $x \in X$ et $\epsilon > 0$, on sait que la boule ouverte
    $$B\left(f(x), \epsilon\right) = \left\{ y \in Y \mid \rho(f(x), y) < \epsilon \right\}$$
    est ouverte. Alors $x \in f^{-1}\left(B(f(x), \epsilon)\right)$ et $f^{-1}\left(B(f(x), \epsilon)\right)$ est ouvert.
    \newline Il existe donc $\delta > 0$ tel que 
    $$w \in f^{-1}\left(B(f(x), \epsilon)\right) \text{ quand } d(x, w) < \delta$$
    c'est a dire que $\rho(f(x), f(w)) < \epsilon$ quand $d(x, w) < \delta$ ce qui implique que $f$ est continue.
  \end{preuve}
\end{theoreme}
Avec ce théorème, on démontre la loi de composition très facilement: si $U$ est ouvert dans $Z$ alors, par continuité de $f$, $f^{-1}(U)$ est ouvert dans $Y$ et, par continuité de $g$, $(f \circ g)^{-1}(U) = g^{-1}\left(f^{-1}(U)\right)$ est ouvert dans $X$. Alors $f \circ g$ est continue.

\subsection{Ensembles fermés dans les espaces métriques}
\begin{definition}
  Soit $x_n$ une suite dans un espace métrique $(X, d)$. Si pour $x \in X$ et $\epsilon >0$ donné, il existe un entier $N \geq 1$ (qui dépend que $\epsilon$) tel que
  $$d(x_n, x) < \epsilon \text{ pour tout } n \geq N$$
  nous disons que $x_n \ra x$ quand $n \ra \infty$ et que $x$ est la limite de la suite $x_n$.
\end{definition}

\begin{lemme}
  Si une suite $x_n$ dans un espace métrique $(X, d)$ a une limite alors cette limite est unique.

  \tcblower
  \begin{preuve}
    Procédons a une preuve par l'absurde. Supposons que $x_n \ra x$ et $x_n \ra y$ avec $x \neq y$. Pour chaque $\epsilon > 0$, on peut trouver $N_1, N_2$ entier tel que
    $$d(x_n, x) < \frac{\epsilon}{2} \text{ pour tout } n \geq N_1$$
    et
    $$d(x_n, y) < \frac{\epsilon}{2} \text{ pour tout } n \geq N_2$$
    En posant $N = \max\{N_1, N_2\}$, on obtient
    $$d(x, y) \leq d(x, x_n) + d(x_n, y) \leq \frac{\epsilon}{2} + \frac{\epsilon}{2} = \epsilon$$
    Comme $\epsilon$ est arbitraire, on a $d(x, y) = 0$ ce qui implique que $x = y$.
  \end{preuve}
\end{lemme}
En utilisant le concept des suites et de leur limites, nous pouvons définir les ensembles fermés.

\begin{definition}
  Soit $(X, d)$ un espace métrique. Un ensemble $F \subseteq X$ est fermé si pour toute suite $x_n \in F$ qui converge vers $x$, $x \in F$.
\end{definition}

% TODO: Finish this!!!

\section{Espaces topologiques}
\subsection{Définitions et exemples}

\begin{definition}
  Soit $X$ un ensemble et $\tau$ une collection de sous-ensembles de $X$ satisfaisant les propriétés suivantes:
  \begin{enumerate}
    \item[$O_1$] $\varnothing, X \in \tau$
    \item[$O_2$] Si $U_\alpha \in \tau$ pour tout $\alpha \in A$, alors $\bigcup\limits_{\alpha \in A} U_\alpha \in \tau$ 
    \item[$O_3$] Si $U_j \in \tau$ pour tout $1 \leq j \leq n$, alors $\bigcap\limits_{j=1}^{n} U_j \in \tau$
  \end{enumerate}
  Nous disons que $\tau$ est une \textbf{topologie} sur $X$ et que $(X, \tau)$ est un \textbf{espace topologique}.
\end{definition}
Si $(X, \tau)$ est un espace topologique, nous appelons ensembles ouverts les élèments de $\tau$.
\newline Prenons un espace métrique $(X, d)$ et posons $\tau_d = \{U \subseteq X \mid U \text{ ouvert par } d\}$. Alors $(X, \tau_d)$ est un espace topologique et $\tau_d$ est appelé la \textbf{\it topologie induite par la métrique} $d$.
\newline Prenons ensuite l'ensemble $X = \{0, 1\}$. On peut munir celui-ci de plusieurs structures d'espace topologique.
\begin{enumerate}
  \item Si on pose $\tau = \left\{ \varnothing, \{0, 1\} \right\}$, alors $\tau$ satisfait les 3 axiomes et est appelé la topologie indiscrète.
  \item Si on pose $\tau = \left\{ \varnothing, \{0\}, \{1\}, \{0, 1\} \right\}$, alors $\tau$ satisfait les 3 axiomes et est appelé la topologie discrète.
\end{enumerate}
% TODO: palenight theme
\begin{lemme}
  Soit $(X, \tau)$ un espace topologique. Si $\tau$ est induite par une métrique $d$, alors pour chaque paire de points distincts $a, b \in X$, il existe des ouverts disjoints $U_a$ et $U_b$ tel que $a \in U_a$, $b \in U_b$.

  \tcblower
  \begin{preuve}
    Soit $r = d(a, b)$ et posons
    $$U_a = B\left(a, \frac{r}{3} \right) \; \text{ et } \; U_b = B\left(b, \frac{r}{3}\right)$$
    Alors, trivialement, $a \in U_a$, $b \in U_b$ et $U_a \cap U_b = \varnothing$.
  \end{preuve}
\end{lemme}
Définissons ensuite ce qu'est un ensemble ferme dans un espace topologique.
\begin{definition}
  Soit $(X, \tau)$ un espace topologique. Un ensemble $A$ dans $X$ est dit fermé si son complément est ouvert.
\end{definition}

\begin{theoreme}
  Si $(X, \tau)$ est un espace topologique alors les affirmations suivantes sont vraies:
  \begin{itemize}
    \item[$(F_1)$] $\varnothing$ et $X$ sont fermés.
    \item[$(F_2)$] Si $F_\alpha$ est fermé pour tout $\alpha \in A$ alors $\bigcap_{\alpha \in A} F_\alpha$ est fermé.
    \item[$(F_3)$] Si $F_j$ est fermé pour tout $1 \leq j \leq n$ alors $\bigcup_{j=1}^n F_j$ est fermé.
  \end{itemize}

  \tcblower
  \begin{preuve}
    $(F_1)$ $\varnothing^{c} = X$ est ouvert donc $\varnothing$ est fermé. De manière analogue, $X^{c} = \varnothing$ est ouvert donc $X$ est fermé.
    \par $(F_2)$ Montrons que $\left(\bigcap_{\alpha \in A} F_\alpha \right)^{c}$ est ouvert. Clairement
    $$\left(\bigcap_{\alpha \in A} F_\alpha \right)^{c} = \bigcup_{\alpha \in A} X \setminus F_\alpha$$
    Par hypothèse, $F_\alpha$ est fermée alors $F_\alpha^{c} = X \setminus F_\alpha$ est ouvert. L'union d'ouverts est ouverte. En conséquence, on déduit que $\bigcap_{\alpha \in A} F_\alpha$ est fermé.
    \par $(F_3)$ De manière analogue, montrons que
    $$\left(\bigcup_{j=1}^n F_j \right)^{c}$$ 
    est ouvert. Clairement,
    $$\left(\bigcup_{j=1}^n F_j \right)^{c} = \bigcap_{j=1}^{n} X \setminus F_j.$$
    On sait aussi que $X \setminus F_j = F_j^{c}$ est ouvert. L'intersection d'ouverts est ouverte. En conclusion, on déduit que $\bigcup_{j=1}^n F_j$ est fermé.
  \end{preuve}
\end{theoreme}

\begin{theoreme}
  Soient $(X, \tau)$ et $(Y, \sigma)$ des espaces topologiques. Une fonction $f: X \ra Y$ est continue si et seulement si $f^{-1}(F)$ est fermé dans $X$ quand $F$ est fermé dans $Y$.
\end{theoreme}

\subsection{Intérieur et fermeture}
\begin{definition}
  Soit $(X, \tau)$ un espace topologique et $A$ un sous-ensemble de $X$. Nous écrivons
  $$\Int(A) = \bigcup \{ U \in \tau : U \subseteq A \} \text{ et } \Cl(A) = \bigcap \{ F \: \mathrm{ ferme} : F \supseteq A \}$$ 
  et nous l'appelons respectivement l'intérieur de $A$ et la fermeture de $A$.
\end{definition}

\begin{lemme}
  Nous avons $(\Cl(A^{c}))^{c} = \Int(A)$ et $(\Int(A^{c}))^{c} = \Cl(A)$.
\end{lemme}

\begin{lemme}
  Soit $(X, \tau)$ un espace topologique et $X \subseteq A$.
  \begin{enumerate}
    \item $\Int(A) = \left\{ x \in A : \exists U \in \tau \text{ avec } x \in U \subseteq A \right\}.$ 
    \item $\Int(A)$ est le plus large ouvert contenu dans $A$, c'est-a-dire $\Int(A)$ est l'unique $V \in \tau$ tel que $V \subseteq A$ et, si $W \in \tau$ et $V \subseteq W \subseteq A$ alors $V = W$.
  \end{enumerate}

  \tcblower
  \begin{preuve}
    1. Ceci est juste l'observation que
    $$\Int(A) = \bigcup \left\{ U \in \tau : U \subseteq A \right\} =\left\{ x \in A : \exists U \in \tau \text{ avec } x \in U \subseteq A \right\} $$
    2. Puisque $\Int(A) = \bigcup \left\{ U \in \tau : U \subseteq A \right\}$ nous savons que $\Int(A) \subseteq A$. Vu que l'union d'ouvert est ouverte, $\Int(A)$ est ouvert.
    \newline Si $W \in \tau$ alors $W \subseteq A$ donc
    $$\Int(A) = \bigcup \left\{ U \in \tau : U \subseteq A \right\} \supseteq W$$
    et si $W \supseteq \Int(A)$ alors $W = \Int(A)$.
  \end{preuve}  
\end{lemme}

\section{Davantage sur les structures topologiques}
\subsection{Homéomorphismes}

\begin{definition}
  Nous disons que deux espaces topologiques $(X, \tau)$ et $(Y, \sigma)$ sont homéomorphes s'il existe une bijection $\theta: X \ra Y$ telle que $\theta$ et $\theta^{-1}$ sont continues. Dans ce cas, nous appelons $\theta$ un homéomorphisme.
\end{definition}
\begin{remarque}
  Homéomorphisme implique équivalence en ce qui concerne la topologie.
\end{remarque}

\begin{lemme}
  Homéomorphisme est une relation d’équivalence sur les espaces topologiques.
  \tcblower
  \begin{preuve}
    On définit la relation d'équivalence de la manière suivante : $(X, \tau) \sim (Y, \sigma)$ si $(X, \tau)$ et $(Y, \sigma)$ sont homéomorphes.
    \begin{enumerate}
      \item Il est clair que $\sim$ est binaire
      \item $\sim$ est réflexive, car il est clair que la fonction identité est un homéomorphisme.
      \item Il est clair que $\sim$ est symétrique parce que si $X \overset{\theta}{\sim} Y$ alors $Y \overset{\theta^{-1}}{\sim}X$
      \item $\sim$ est transitive
    \end{enumerate}
  \end{preuve}
\end{lemme}
Pour donner un exemple, $]-1, 1[$ est homéomorphe à $\R$. En fait, on peut définir une fonction $f: ]-1, 1[ \ra \R$ par
$$f(x) = \tan\left(\frac{\pi x}{2} \right)$$ 
qui est une bijection et a une réciproque continue $g: \R \ra ]-1, 1[$ donnée par
$$g(y) = \frac{2}{\pi}\arctan(x)$$

\begin{remarque}
  $(\R^2, d_\text{euc})$ et $(\R^2, d_\infty)$ ne sont pas isométriques, mais $(\R^2, \tau_\text{euc})$ et $(\R^2, \tau_\infty)$ sont homéomorphes ! 
\end{remarque}

\begin{theoreme}
  Soit $f: (X, \tau) \ra (Y, \sigma)$ une bijection continue. Les conditions suivantes sont equivalentes:
  \begin{enumerate}
    \item $f(U)$ est ouvert dans  $Y$ si $U$ est ouvert dans $X$.
    \item  $f(F)$ est fermé dans $Y$ si $F$ est fermé dans $X$.
    \item $f$ est un homéomorphisme.
  \end{enumerate}

  \tcblower
  \begin{preuve}
    Commençons par prouver que 1 implique 2. Soit $F \subseteq X$ fermé. Alors $F^{c}$ est ouvert. On a aussi
    \begin{align*}
      f(F^{c}) &= \left\{ f(x) \in Y \mid x \not \in F \right\} \\
               &= \left\{ f(x) \in Y \mid f(x) \not \in f(F) \right\}
    \end{align*}
    La deuxième égalité découle du fait que $f$ est une bijection continue. De plus, on a
    $$f(F)^{c} = \left\{ f(x) \in Y \mid f(x) \not \in f(F) \right\}$$
    Par définition, $F^{c}$ est ouvert et par hypothèse, on sait que $f(F^{c})$ l'est aussi. Cela implique que $f(F)^{c}$ est ouvert donc $f(F)$ est fermé.
    \par Prouvons ensuite que 2 implique 3. Posons $g = f^{-1}$ et $F \subseteq X$ fermé alors $g^{-1}(F) = f(F)$ qui est fermé par hypothèse. Ceci implique que $g$ est continue. On trouve donc que $f$ est $f^{-1}$ sont réciproques et toutes les 2 continues ce qui implique que $f$ est un homéomorphisme.
    \par Le fait que 3 implique 1 se déduit trivialement.
  \end{preuve}
\end{theoreme}

\subsection{Sous-espaces topologiques}
Cette section est motivée par le fait de vouloir construire des espaces topologiques à partir d'autres.

\begin{lemme}
  Soit $X$ un espace et, soit $\mathcal{H}$ une collection de sous-ensembles de $X$. Alors, il existe une topologie unique  $\tau_\mathcal{H}$ telle que
  \begin{enumerate}
    \item $\tau_\mathcal{H} \supseteq \mathcal{H}$.
    \item Si $\tau$ est une topologie avec $\tau \supseteq \mathcal{H}$ alors $\tau \supseteq \tau_\mathcal{H}$.
  \end{enumerate}

  \tcblower
  \begin{preuve}
    Prouvons d'abord l'unicité. Soit $\tau_\mathcal{H}$ et $\tau_\mathcal{H}'$ deux topologies qui satisfont les conditions. Puisque  $\tau_\mathcal{H} \supseteq \mathcal{H}$, on a $\tau_\mathcal{H}' \subseteq \tau_\mathcal{H}$. En échangeant les rôles, on obtient $\tau_\mathcal{H} \subseteq \tau_\mathcal{H}'$. On en conclut que $\tau_\mathcal{H} \tau_\mathcal{H}'$.
  \end{preuve}
\end{lemme}

\begin{lemme}
  Soit $A$ non-vide, $(X_\alpha, \tau_\alpha)$ des espaces topologiques et $f_\alpha: X \ra X_\alpha$ des applications avec  $\alpha \in A$. Alors, il existe une plus petite topologie $\tau$ sur $X$ pour laquelle $f_\alpha$ sont continues.

  \tcblower
  \begin{preuve}
    Posons $$\H = \left\{ f_\alpha^{-1}(U) \mid \alpha \in A, U \in \tau_\alpha \right\}$$. On sait que les application $f_\alpha$ sont continues pour $\tau$ si  $f_\alpha^{-1}(U) \in \tau$. Posons donc $\tau = \tau_\H$. Alors, on obtient que  $\forall f_\alpha, \forall U \in \tau_\alpha, f_\alpha^{-1}(U) \in \tau_\H$, car $\H \subseteq \tau_\H$. De plus, tout $f_\alpha^{-1}(U) \in X$ donc $\tau_\H$ est une topologie sur  $X$ telle que  $\forall \alpha \in A$, $f_\alpha$ est continue.
  \end{preuve}
\end{lemme}

Si $Y \subseteq X$ alors l'application inclusion $j: Y \ra X$ est définie par  $j(y) = y$ pour tout  $y \in Y$.

\begin{definition}
  Si $(X, \tau)$ est un espace topologique et $Y \subseteq X$, alors la topologie de sous-espace $\tau_Y$ sur $Y$ induite par  $\tau$ est la plus petite topologie sur $Y$ pour laquelle l'application inclusion est continue. 
  \par On dit alors que $Y$ est un sous-espace topologique de $X$.
\end{definition}
Caractérisons la topologie de sous-espace.
\begin{lemme}
  Soit $(X, \tau)$ un espace topologique et $Y \subseteq X$. Alors la topologie de sous-espace $\tau_Y$ sur  $Y$ est
  $$\tau_Y = \left\{ Y \cap U \mid U \in \tau \right\}.$$

  \tcblower
  \begin{preuve}
    Posons $$\theta = \left\{ Y \cap U \mid U \in \tau \right\}.$$ Puisque $j: (Y, \tau_Y) \ra (X, \tau)$ est continue par hypothèse, on a que si  $U \in \tau$ alors $j^{-1}(U) \in \tau_Y$. Or,
    \begin{align*}
      j^{-1}(U) &= \left\{ y \in Y \mid j(y) \in U \right\} \\
                &= \left\{ y \in Y \mid y \in U \right\} \\
                &= Y \cap U \in \tau_Y.
    \end{align*}
    Donc $\tau_Y$ est la plus petite topologie sur $Y$ contenant $\theta$. Montrer que $\theta$ est une topologie se fait trivialement.
  \end{preuve}
\end{lemme}

\subsection{Produits d'espaces topologiques}
\begin{definition}
  Si $(X, \tau)$ et  $(Y, \sigma)$ sont des espaces topologiques, alors la topologie produit $\mu$ sur  $X \times Y$ est la plus petite topologie sur $X \times Y$ pour laquelle les applications de projections
  \begin{align*}
    \pi_X: X \times Y &\longrightarrow X \\
    (x, y)&\longmapsto x
  \end{align*}
  et
  \begin{align*}
    \pi_Y: X \times Y &\longrightarrow Y \\
    (x, y)&\longmapsto y
  \end{align*}
  sont continues.
\end{definition}

\begin{lemme}
  Soient $(X, \tau)$ et $(Y, \sigma)$ des espaces topologiques et $\mu$ la topologie produit sur $X \times Y$. Alors  $O \in \mu$ si et seulement si, pour $(x, y) \in O$ donnes, nous pouvons trouver $U \in \tau$ et $V \in \sigma$ tels que
  $$(x, y) \in U \times V \subseteq O.$$
\end{lemme}
Pour reconnaître des espaces comme homéomorphes à des produits d’autres espaces, il sera utile d’avoir une description simple pour les applications vers les espaces produit.
\begin{propo}
  Soient $(X, \tau), (Y, \sigma)$ et $(Z, \rho)$ des espaces topologiques. Une application continue  $f: X \ra Y \times Z$ correspond à une paire de fonctions continue  $f_Y: X \ra Y$ et  $f_Z: X \ra Z$.

  \tcblower
  \begin{preuve}
    % TODO: proof
  \end{preuve}
\end{propo}

\begin{lemme}
  Soient $\tau_1$ et  $\tau_2$ deux topologies sur le même espace $X$. Nous avons  $\tau_1 \subseteq \tau_2$ si et seulement si, pour $x \in U \in \tau_1$ donné, nous pouvons trouver $V \in \tau_2$ tel que $x \in V \subseteq  U$.
  \par Nous avons $\tau_1 = \tau_2$ si et seulement si, $\tau_1 \subseteq \tau_2$ et $\tau_2\subseteq \tau_1$.
\end{lemme}

\subsection{Topologie quotient}
Si $\sim$ est une relation d'équivalence sur un ensemble $X$, nous savons qu'elle donne l'origine a des classes d'équivalences
$$[x] = \left\{ y \in X \mid y \sim x \right\}.$$
Il existe une application naturelle $q$ des $X$ vers l'ensemble des classes d'équivalences $X/\sim$ qui est donnée par $q(x) = [x]$.

\begin{lemme}
  Soit $(X, \tau)$ un espace topologique et $Y$ un ensemble. Si $f: X \ra Y$ est une application et nous écrivons
  $$\sigma = \left\{ U \subseteq Y \mid f^{-1}(U) \in \tau \right\}$$ 
  alors $\sigma$ est une topologie sur $Y$ telle que
  \begin{enumerate}
    \item $f: X \ra Y$ est continue.
    \item Si $\theta$ est une topologie sur $Y$ avec $f: (X, \tau) \ra (Y, \sigma)$ continue, alors $\theta \subseteq \sigma$.
  \end{enumerate}

  \tcblower
  \begin{preuve}
    1. ??? % TODO: proof
    \par 2. Montrons que $\sigma$ est une topologie.
    \begin{enumerate}
      \item $f^{-1}(\varnothing) = \varnothing \in \tau \implies \varnothing \in \sigma$. Ensuite, $f^{-1}(X) = X \in \tau \implies X \in \sigma$.
      \item Si $\forall \alpha \in A, U_\alpha \in \sigma, f^{-1}(U_\alpha) \in \tau$ et donc
        $$f^{-1}\left(\bigcup_{\alpha \in A} U_\alpha \right) = \bigcup_{\alpha \in A} f^{-1}\left( U_\alpha \right) \in \tau \implies \bigcup_{\alpha \in A} U_\alpha \in \sigma.$$ 
      \item Si pour $j \in \left\{ 1, \dots, n \right\}, U_j \in \sigma$ on a que $f^{-1}(U_j) \in \tau$. Donc,
        $$f^{-1}\left( \bigcap_{j=1}^nU_j \right) = \bigcap_{j=1}^n f^{-1}(U_j) \in \tau \implies \bigcap_{j=1}^n U_j \in \sigma.$$
    \end{enumerate}
  \end{preuve}
\end{lemme}

\begin{definition}\label{def:qutop}
  Soit $(X, \tau)$ un espace topologique et $\sim$ une relation d'équivalence sur $X$. Écrivons  $q$ pour l'application de $X$ vers l'espace $X/\sim$ donnée par $q(x) = [x]$. La topologie quotient $\sigma$ est la topologie la plus large sur $X/\sim$ pour laquelle $q$ est continue, c'est-à-dire
  $$\sigma = \left\{ U \subseteq X/\sim \mid q^{-1}(U) \in \tau \right\}.$$
\end{definition}

\begin{lemme}
  En utilisant les suppositions et notations de la définition \ref{def:qutop}, la topologie quotient consiste à des ensembles $U$ tels que
  $$\bigcup_{[x] \in U} [x] \in \tau.$$

  \tcblower
  \begin{preuve}
    $$\sigma = \left\{ U \in X/\sim \mid q^{-1}(U) \in \tau \right\}$$ 
    Effectuons le calcul
     \begin{align*}
       q^{-1}(U) &= \left\{ x \in X \mid [x] \in U \right\} \\
                 &= \bigcup_{[x] \in U} [x]
    \end{align*}
    Ceci implique que
    $$\sigma = \left\{ U \in X/\sim \mid \bigcup_{[x] \in U} [x] \in \tau \right\}.$$
  \end{preuve}
\end{lemme}

\section{Espaces de Hausdorff}
\begin{definition}
  Un espace topologique $(X, \tau)$ est dit d'Hausdorff si $\forall x, y \in X$ avec $x \not = y$, il existe  $U_x, U_y \in \tau$ tels que $x \in U_x$, $y \in U_y$ et $U_x \cap U_y = \varnothing$.
\end{definition}
Une remarque importante à prendre en compte est que les espaces métriques sont Hausdorff. Si $(X, d)$ est un espace métrique alors la topologie induite par cette métrique est Hausdorff.
\begin{definition}
  Soit $(X, \tau)$ un espace topologique et $x \in U \in \tau$. On dit que $U$ est un voisinage ouvert de $x$. 
\end{definition}

\begin{lemme}\label{lem:ouv}
  Si $(X, \tau)$ est un espace topologique, alors un sous-ensemble $A \subseteq X$ est ouvert si et seulement si tout point de $A$ a un voisinage ouvert $U \subseteq A$.

  \tcblower
  \begin{preuve}
    Prouvons d'abord l'implication vers la droite. Si $A$ est ouvert, c'est-à-dire $A \in \tau$ alors $\forall x \in A$, $x \in A \subseteq A$.
    \par Traitons ensuite l'implication vers la gauche. Par hypothèse, $\forall x \in A$, $x$ possède un voisinage ouvert $U_x \subseteq A$. Cela implique que
    $$\bigcup_{x\in A} U_x \subseteq A.$$
    Puisque tous les points de $A$ admettent un voisinage ouvert, on sait aussi que
    $$A \subseteq \bigcup_{x \in A} U_x.$$
    On conclut que $$A = \bigcup_{x \in A} U_x$$.
  \end{preuve}
\end{lemme}

\begin{propo}\label{prop:singl}
  Si $(X, \tau)$ est un espace de Hausdorff alors les singletons $\left\{ x \right\}$ sont fermés.

  \tcblower
  \begin{preuve}
    Montrons que $A = X \setminus \left\{ x \right\}$ est ouvert.
    \par Soit $y \in A$, c'est-à-dire $y \not = x$. Par hypothèse, il existe  $U, V \in \tau$ tels que $x \in U$, $y \in V$, et $U \cap V = \varnothing$. Puisque $x \not \in V$, on sait que $\forall y \in A$, $y \in V \subseteq A$. On conclut par le lemme \ref{lem:ouv} que $A$ est ouvert et donc que $X \setminus A = \left\{ x \right\}$ est fermé.
  \end{preuve}
\end{propo}

\begin{remarque}
  La réciproque de la proposition \ref{propo:singl} est fausse.
\end{remarque}

\begin{propo}
  Soit $(X, \tau)$ un espace topologique. Si $(X, \tau)$ est Hausdorff alors $Y \subseteq X$ avec la topologie de sous-espace l'est aussi.

  \tcblower
  \begin{preuve}
    Soit $\tau_Y$ la topologie de sous-espace de $Y$ sur $X$. Soit $x, y \in Y$ tels que $x \not = y$. Alors $x, y \in X$. Par hypothèse, $\exists U, V \in \tau$ tels que $x \in U$, $y \in V$ et $U \cap V = \varnothing$.
    \par Posons $\tilde{U} = U \cap Y$ et $\tilde{V} = V \cap Y$. Par définition de  $\tau_Y$, on sait que  $\tilde{U}, \tilde{V} \in \tau_Y$. Puisque $x, y \in Y$, on sait que $x \in \tilde{U} \subseteq U$ et $y \in \tilde{V} \subseteq V$. En effectuant l'intersection
    $$\tilde{V} \cap \tilde{U} = (Y \cap V) \cap (Y \cap U) = \varnothing$$
    ce qui implique que $(Y, \tau_Y)$ est Hausdorff.
  \end{preuve}
\end{propo}
 
\begin{propo}
  Si $(X, \tau)$ et $(Y, \sigma)$ sont des espaces d'Hausdorff alors $X \times Y$ avec la topologie produit l'est aussi.

  \tcblower
  \begin{preuve}
    Soient $a, b \in X$ et $c, d \in Y$ tels que $(a, b) \not = (c, d)$. Si $a \not = c$ et $b \not = d$,  $\exists U, V \in \tau$ tels que $a \in U$, $c \in V$ et $U \cap V = \varnothing$. De plus, $\exists \tilde{U}, \tilde{V} \in \sigma$ tels que $b \in \tilde{U}$, $d \in \tilde{V}$ et $\tilde{U} \cap \tilde{V} = \varnothing$.
  \end{preuve}
\end{propo}

\begin{propo}
  Soient $(X, \tau)$ et $(Y, \sigma)$ des espaces topologiques. Si $f: X \ra Y$ est continue et injective et $Y$ est Hausdorff alors $X$ l'est aussi.

  \tcblower
  \begin{preuve}
    Soient $x, y \in X$ avec $x \not = y$. Puisque $f$ est injective, on sait que $f(x) \not = f(y)$.
    \par Vu que  $Y$ est Hausdorff, $\exists U, V \in \sigma$ tels que $f(x) \in U$, $f(y) \in V$ et $U \cap V = \varnothing$.
    \par Par continuité de $f$, on a que  $f^{-1}(U), f^{-1}(V) \in \tau$. Comme $f$ est injective, $f^{-1}(U) \cap f^{-1}(V) = \varnothing$ et vu que $x \in f^{-1}(U)$ et $y \in f^{-1}(V)$, l'espace $(X, \tau)$ est Hausdorff.
  \end{preuve}
\end{propo}

Nous voyons que tout le sous-espace d’un espace d'Hausdorff l’est aussi. En particulier, tous les sous-espaces de $\R^n$ sont Hausdorff ! Néanmoins, il existe des espaces topologiques qui ne sont pas Hausdorff. La topologie indiscrète est un exemple simple, mais il y en a d’autres.

\section{Compacité}
\subsection{Espaces compactes}

\begin{definition}
  Un espace topologique $(X, \tau)$ est dit \textbf{compact} si, pour chaque collection $\left\{ U_\alpha \right\}_{\alpha \in A}$ d'ensembles ouverts avec $\bigcup\limits_{\alpha \in A}U_\alpha = X$, nous pouvons trouver une sous-collection finie $U_{\alpha(1)}, \dots, U_{\alpha(n)}$ avec  $\alpha(j) \in A$ et $1 \leq j \leq n$ tel que  $\bigcup\limits_{j=1}^n U_{\alpha(j)} = X$.
  \par Nous disons que $\left\{ U_\alpha \right\}_{\alpha \in A}$ tel que $\bigcup\limits_{\alpha \in A} U_\alpha = X$ est un \textbf{recouvrement} de $X$ (par des ouverts).
\end{definition}

Par exemple, la droite réelle $\R$ n'est pas compacte, car le recouvrement de  $\R$ par les ouverts
$$\left\{\; ]n, n+2[ \; \mid n \in \Z \; \right\}$$ 
ne contient pas une sous-collection finie recouvrant $\R$.

\begin{remarque}
  Tout espace $X$ contenant un nombre fini de points est forcément compact, car dans ce cas tous les recouvrements de $X$ sont finis.
\end{remarque}

\begin{definition}
  Si $(X, \tau)$ est un espace topologique, alors un sous-ensemble $Y \subseteq X$ est dit compact si la topologie de sous-espace sur $Y$ est compacte.
\end{definition}

\begin{lemme}
  Un sous-ensemble $Y$ d'un espace topologique $(X, \tau)$ est dit compact si, pour chaque collection $\left\{ U_\alpha \right\}_{\alpha \in A}$ d'ensembles ouverts avec $\bigcup\limits_{\alpha \in A} U_\alpha \supseteq Y$, nous pouvons trouver une sous-collection finie $U_{\alpha(1)}, \dots, U_{\alpha(n)}$ avec $\alpha(j) \in A$ et $1 \leq j \leq n$, tel que $\bigcup\limits_{j=1}^n U_{\alpha(j)} \supseteq Y$.

  \tcblower
  \begin{preuve}
    Si $Y$ est compact, par définition, il existe une collection $\left\{ V_\alpha \right\}_{\alpha \in A}$ telle que $V_\alpha \in \tau_Y$ pour tout $\alpha \in A$ avec $$Y = \bigcup_{\alpha \in A} V_\alpha.$$
    Vu que $V_\alpha \in \tau_Y, \exists U_\alpha \in \tau$ tel que $V_\alpha = Y \cap U_\alpha$. On sait que  $V_\alpha \subseteq U_\alpha$ et donc la collection $\left\{ U_\alpha \right\}_{\alpha \in A}$ est un recouvrement de $Y$ dans $X$, c'est-à-dire  $$Y \subseteq \bigcup_{\alpha \in A} U_\alpha.$$
    Puisque $Y$ est compact, $\exists \alpha(1), \dots, \alpha(n) \in A$ tel que
    $$Y = \bigcup_{i=1}^n V_{\alpha(i)} = \bigcup_{i=1}^n \left( Y \cap U_{\alpha(i)} \right).$$
    Étant donné que pour $1 \leq i \leq n$, on a $Y \cap U_{\alpha(i)} \subseteq U_{\alpha(i)}$ et donc
    $$Y = \bigcup_{i=1}^n \left( Y \cap U_{\alpha(i)} \right) \subseteq \bigcup_{i=1}^n U_{\alpha(i)}.$$
    Ainsi, si $Y$ est compact, pour tout recouvrement $\left\{ U_\alpha \right\}_{\alpha \in A}$ de $Y$ dans $X$, il existe un sous-recouvrement fini $\left\{ U_{\alpha(i)} \right\}_{1 \leq i \leq n}$ de $Y$ dans $X$.
  \end{preuve}
\end{lemme}

L'idée est qu'un ensemble est compact si chaque fois qu'il est recouvert par des ouverts, il est recouvert par un nombre fini d'entre eux.
\par Nous disons qu’un sous-ensemble d’un espace métrique est borné s’il est contenu dans une boule ouverte avec un rayon fini.

\begin{propo}
  Soit $\R$ muni de la topologie usuelle. L'intervalle fermé et borné $[a, b]$ est compact.

  \tcblower
  \begin{preuve}
    Si $a=b$ le cas est trivial. Supposons $a < b$ et prenons un recouvrement ouvert
    $$[a, b] \subseteq \bigcup_{i \in I} \mathcal{U}_i.$$
    On a en particulier que c'est un recouvrement de $[a, x]$ pour tout $x \in [a, b]$. Posons $S$ l'ensemble de tous les $x\in [a, b]$ tels que $[a, x]$ admet un sous-recouvrement fini de $\bigcup\limits_{i \in I} \U_i$. Il existe alors $i_0 \in I$ tel que $a \in \U_{i_0}$, donc $a \in S$.
    On déduit alors que $S$ est un sous-ensemble non vide de $\R$ borné par $b$. On peut poser
    $$x_0 \coloneq \sup S \in [a, b].$$
    Ensuite, montrons par contradiction que $x_0 = b$. Supposons que $x_0 < b$ et notons que $x_0 > a$. En effet, il existe  $i_0 \in I$ et $\epsilon > 0$ tel que  $[a, a+\varepsilon] \subseteq \U_{i_0}$, et donc $x_0 \geq a + \varepsilon$.
    \par Prenons $i_0 \in I$ tel que $x_0 \in \U_{i_0}$ et $\varepsilon > 0$ tel que  $a \leq x_0 - \varepsilon < x_0 < x_0 + \varepsilon \leq b$. Alors
    $$[x_0 - \varepsilon, x_0 + \varepsilon] \subseteq \U_{i_0}.$$
    Puisque $x_0 - \varepsilon$ n'est pas un supremum de $S$, il existe $x_0 - \varepsilon \leq x_1 \leq x_0$ tel que $x_1 \in S$. De telle manière, l'intervalle $[a, x_1]$ admet un recouvrement fini, c'est-à-dire
    $$[a, x_1] \subseteq \bigcup_{j=1}^n \U_{i_j}.$$
    Mais alors, comme $x_0 - \varepsilon \leq x_1 \leq x_0$ et comme $[x_0 - \varepsilon, x_0+\varepsilon] \subseteq U_{i_0}$, on a que
    $$[a, x_0+\varepsilon] \subseteq \bigcup_{j=1}^n U_{i_j} \cup U_{i_0}.$$
    Il suit que $x_0 + \varepsilon \in S$ ce qui contredit $x_0 = \sup S$. Ainsi, $\sup S = b$ et un argument analogue montre que  $b \in S$. Ceci implique qu'il existe un recouvrement fini
    $$[a, b] \subseteq \bigcup_{j=1}^n \U_{i_j}.$$
  \end{preuve}
\end{propo}

\begin{theoreme}[Heine-Borel]\label{thm:borel}
  Un sous-espace $T$ de $\R^n$ (muni de la topologie usuelle) est compact si et seulement s'il est fermé (comme un sous-ensemble) et borné.
\end{theoreme}
Nous allons déduire la preuve de ce théorème comme conséquences de quelques théorèmes à suivre.

\begin{theoreme}
  Un sous-ensemble fermé d'un ensemble compact est compact. Plus précisément, soit $(X, \tau)$ un espace topologique. Si $E \subseteq X$ est compact et $F$ est fermé dans une topologie donnée alors, si $F \subseteq  E$ nous avons que $F$ est aussi compact.

  \tcblower
  \begin{preuve}
    Soit $\left\{ \U_\alpha \right\}_{\alpha \in A}$ un recouvrement de $F$. Puisque $X \setminus F \in \tau$, on a que
    $$\bigcup_{\alpha \in A} \U_\alpha \cup (X \setminus F) = X \supseteq E$$
    qui est un recouvrement de $E$.
    \par Vu que $E$ est compact, $\exists \alpha(j) \in  A$ avec $1 \leq j \leq n$ tels que
    $$E \subseteq (X \setminus F) \cup \bigcup_{j=1}^n \U_{\alpha(j)}.$$
    Comme $X \setminus F \cap F = \varnothing$ et $F \subseteq E$, on a
    $$F \subseteq \bigcup_{j=1}^n \U_{\alpha(j)}$$
    ce qui implique que $F$ est compact.
  \end{preuve}
\end{theoreme}

La conséquence de ce théorème par rapport au théorème \ref{thm:borel} est la suivante. Si $A \subseteq \R^n$ est fermé et borné, alors $A$ est compact. La démonstration est assez simple. Comme $A$ est borné,
$$A \subseteq B(0, R) \subseteq [-R, R]^n.$$ 
Puisque $[-R, R]^n$ est compact, $A$ est fermé et $A \subseteq [-R, R]^n$, on a que $A$ est compact.

\begin{propo}
  Si $A \subseteq \R^n$ est compact, alors $A$ est borné.

  \tcblower
  \begin{preuve}
    Soit $A \subseteq \R^n$ compact, alors
    $$A \subseteq \bigcup_{R>0} B(0, R) = \R^n.$$
    Par compacité, $\exists R_1, \dots, R_n$ tel que
    $$A \subseteq \bigcup_{i=1}^n B(0, R_i) = B(0, \bar{R}) \text{ avec } \bar{R} = \max R_i.$$
    On déduit que $A$ est borné.
  \end{preuve}
\end{propo}
La conséquence pour le théorème \ref{thm:borel} est immédiate.

\begin{theoreme}
  Si $(X, \tau)$ est un espace d'Hausdorff, alors tout sous-ensemble $K \subseteq X$ compact est fermé.

  \tcblower
  \begin{preuve}
    Soient $c \in K$ et $x \in X\setminus K$, alors $x \not = c$. Comme  $X$ est Hausdorff, $\exists U_x, U_c \in \tau$ tel que $x \in U_x$, $c \in U_c$ et $U_x \cap U_c = \varnothing$.
    \par Comme
    $$K \subseteq \bigcup_{c \in K} \left\{ c \right\} \subseteq \bigcup_{c \in K} U_c, $$
    on a que $\bigcup\limits_{c \in K} U_c$ est un recouvrement ouvert de $K$. Par compacité de $K$,  $\exists c(1), \dots, c(n)$ tel que $\bigcup\limits_{j=1}^n U_{c(j)}$ est un recouvrement fini de  $K$.
    \par Puisque  $\forall x \in X \setminus K, U_x \cap U_c = \varnothing$, on a que
    $$U_x \cap \bigcup_{j=1}^n U_{c(j)} = \varnothing$$
    ce qui implique que $U_x \cap K = \varnothing$.
    \par Comme $\forall x \in X \setminus K$, $\exists U_x$ avec $U_x \cap K = \varnothing$, $x$ possède un voisinage ouvert: $x \in U_x \subseteq X \setminus K$. Ceci implique que $X \setminus K$ est ouvert donc $K$ est fermé.
  \end{preuve}
\end{theoreme}

La conséquence par rapport au théorème \ref{thm:borel} est la suivante: Si $A \subseteq \R^n$ est compact, alors $A$ est fermé. La démonstration est la suivante. Puisque $\R^n$ muni de la topologie usuelle est Hausdorff et étant donné que $A$ est compact, on a par le théorème précédent que $A$ est fermé.

\begin{theoreme}\label{thm:comp}
  Soient $(X, \tau)$ et $(Y, \sigma)$ des espaces topologiques et $f: X \ra Y$ une fonction continue. Si $K$ est un sous-ensemble compact de $X$ alors $f(K)$ est un sous-ensemble compact de $Y$.

  \tcblower
  \begin{preuve}
    Soit $A$ un ensemble et soient $U_\alpha \in \sigma$ avec $\alpha \in A$ tels que $$f(K) \subseteq \bigcup\limits_{\alpha \in A} U_\alpha.$$
    Alors
    $$\bigcup_{\alpha \in A} f^{-1}(U_\alpha) = f^{-1}\left( \bigcup_{\alpha \in  A} U_\alpha \right) \supseteq K.$$ 
    Par continuité de $f$,
     $$f^{-1}\left( \bigcup_{\alpha \in  A} U_\alpha \right) \in \tau.$$
     Par compacité de $K$,  $\exists \alpha(1), \dots, \alpha(n) \in A$ tels que
     $$K \subseteq \bigcup_{j=1}^n f^{-1}(U_{\alpha(j)}) \in \tau.$$
     Donc,
     $$\bigcup_{j=1}^n U_{\alpha(j)} \supseteq f\left( f^{-1}\left( \bigcup_{j=1}^n U_{\alpha(j)} \right) \right) \supseteq K.$$
     Ceci implique que $f(K)$ est compact.
  \end{preuve}
\end{theoreme}

\begin{cor}[Propriété topologique]
  Si $(X, \tau)$ et $(Y, \sigma)$ sont homéomorphes, alors $(X, \tau)$ est compact si et seulement si $(Y, \sigma)$ l'est aussi.

  \tcblower
  \begin{preuve}
    Il existe $f: X \ra Y$ continue et bijective, avec $y = f(x)$ et  $x = f^{-1}(y)$. Par le théorème précédent, si $X$ est compact alors $Y$ est compact et si $Y$ est compact alors $X$ est compact.
  \end{preuve}
\end{cor}

Le théorème \ref{thm:comp} nous donne une propriété agréable pour la topologie quotient.

\begin{cor}
  Soit $(X, \tau)$ un espace topologique compact et $\sim$ une relation d'équivalence sur $X$. Alors la topologie quotient  $X /\sim$ est compacte.

  \tcblower
  \begin{preuve}
    Par définition d'espace topologique quotient,
    \begin{align*}
      q: X &\lra X / \sim \\
      x &\longmapsto [x]
    \end{align*}
    est continue donc la preuve découle du théorème \ref{thm:comp}.
  \end{preuve}
\end{cor}

% TODO: finish this

\begin{theoreme}\label{thm:hom}
  Soit $(X, \tau)$ un espace topologique compact et $(Y, \sigma)$ un espace topologique d'Hausdorff. Si $f: X \ra Y$ est une bijection continue, alors $f$ est un homéomorphisme.

  \tcblower
  \begin{preuve}
    Par bijectivité de $f$ on a que, pour  $U \in \tau$,
    $$\left(f^{-1} \right)^{-1}(U) = f(U) = Y \setminus f(X \setminus U).$$
    Comme $X$ est compact, $X \setminus U$ est compact car fermé.
    \par Comme $Y$ est Hausdorff, $f(X \setminus U)$ est fermé car compact.
    \par Il suit que  $f(U)$ est ouvert ce qui implique que $f^{-1}$ est continue et donc $f$ est un homéomorphisme.
  \end{preuve}
\end{theoreme}

\begin{lemme}\label{lem:id}
  Soient $\tau_1$ et $\tau_2$ des topologies sur un ensemble $X$. L'application identité
  $$\id: (X, \tau_1) \lra (X, \tau_2),$$
  donnée par $\id(x) = x$ est continue si et seulement si $\tau_1 \supseteq \tau_2$.

  \tcblower
  \begin{preuve}
    Montrons d'abord l'implication vers la droite. Soit $U \in \tau_2$. Par continuité de $\id$, on sait que $\id^{-1}(U) = U \in \tau_2$ donc $\tau_2 \subseteq \tau_1$.
    \par Pour l'implication vers la gauche, supposons que $\tau_2 \subseteq \tau_1$. Comme $\forall U \in \tau_2$, on a que $U \in \tau_1$, donc
    $$\id^{-1}(U) = U \in \tau_1,$$
    ce qui implique que $\id$ est continue.
  \end{preuve}
\end{lemme}

\begin{theoreme}
  Soient $\tau_1$ et $\tau_2$ des topologies sur le même ensemble $X$.
  \begin{enumerate}
    \item Si $\tau_1 \supseteq \tau_2$ et $\tau_1$ est compact, alors $\tau_2$ l'est aussi.
    \item Si $\tau_1 \supseteq \tau_2$ et $\tau_2$ est Hausdorff, alors $\tau_1$ l'est aussi.
    \item Si $\tau_1 \supseteq \tau_2$ et $\tau_1$ est compact et $\tau_2$ est Hausdorff, alors $\tau_1 = \tau_2$.
  \end{enumerate}

  \tcblower
  \begin{preuve}
    1. Comme $\tau_1 \supseteq \tau_2$, par le théorème \ref{lem:id} on sait que $\id: (X, \tau_1) \ra (X, \tau_2)$ est continue. De plus, par le théorème \ref{thm:comp}, puisque $\tau_1$ est compact, $\forall U \in \tau_1$, $\id(U) = U \in \tau_2$ est compact. Donc, $\forall U \in \tau_2$, $U$ est compact donc $(X, \tau_2)$ est compact.
    \par 2. Comme $\tau_1 \supseteq \tau_2$ et $\tau_2$ est Hausdorff, $\forall x, y \in X, \exists U, V \in \tau_2$ tels que $x \in U, y \in V$ et $U \cap V = \varnothing$. Comme $\id$ est continue
    $$\id^{-1}(U \in \tau_2) = U \in \tau_1 \; \text{ et } \; \id^{-1}(V \in \tau_2) = V \in \tau_1.$$
    On a donc que $x\in U, y \in V$ et $U \cap V = \varnothing$ avec $U, V \in \tau_1$ ce qui implique que $(X, \tau_1)$ est Hausdorff.
    \par 3. Comme $\tau_1 \supseteq \tau_2$, l'application $\id: (X, \tau_1) \ra (X, \tau_2)$ est continue. Par le théorème \ref{thm:hom} on sait aussi que $\id$ est un homéomorphisme donc $\id^{-1}: (X, \tau_2) \ra (X, \tau_1)$ est continue. Ainsi,
    $$\forall U \in \tau_1, \left( \id^{-1} \right)^{-1} (U) = U \in \tau_2$$
    donc $\tau_1 \subseteq \tau_2$. Comme $\tau_1 \subseteq \tau_2$ et $\tau_1 \supseteq \tau_2$, on a que $\tau_1 = \tau_2$.
  \end{preuve}
\end{theoreme}

\subsection{Produits d'espaces compacts}
\begin{theoreme}[Tychonoff]
  Le produit d'espaces compacts est compact.
\end{theoreme}
Le theoreme signifie que si $(X, \tau)$ et $(Y, \sigma)$ sont des espaces topologiques compacts et $\mu$ est la topologie produit alors $(X \times Y, \mu)$ est compact.

\begin{theoreme}
  Soient $(X, \tau)$ et  $(Y, \sigma)$ des espaces topologiques compacts et soit $\mu$ la topologie produit. Si $K \subseteq X$ et $L \subseteq Y$ sont compacts alors $K \times L$ est compact dans $\mu$.

  \tcblower
  \begin{preuve}
    Établissons d'abord les hypothèses.
    \par $K$ est compact sur $\tau$ si et seulement si pour tout $\left\{ U_\alpha \right\}_{\alpha \in A}$ avec $U_\alpha \in \tau$ et $K \subseteq \bigcup\limits_{\alpha \in A} U_\alpha$, il existe $\alpha(j) \in A$, $1 \leq j \leq n$ tel que
    $$\bigcup_{\alpha(j) \in A} U_{\alpha(j)} \supseteq K.$$
    De manière similaire, $L$ est compact sur $\sigma$ si et seulement si pour tout $\left\{ V_\beta \right\}_{\beta \in B}$ avec $V_\beta \in \tau$ et $L \subseteq \bigcup\limits_{\beta \in B} V_\beta$, il existe $\beta(j) \in B$, $1 \leq j \leq n$ tel que
    $$\bigcup_{\beta(j) \in B} V_{\beta(j)} \supseteq L.$$
    % TODO: ???
  \end{preuve}
\end{theoreme}

\subsection{Compacité dans les espaces metriques}
Nous disons qu'un espace métrique est compact si la topologie induite par la métrique est compacte.

\begin{definition}
  Un espace métrique $(X, d)$ est dit \textbf{séquentiellement compact} si toute suite sur $X$ possède au moins une sous-suite convergente.
\end{definition}
Par exemple, $[a, b]$ muni de la topologie usuelle avec $a, b \in \R$ est séquentiellement compact.
\par $]0, 1]$ n'est pas séquentiellement compact. Prenons la suite $\left( \frac{1}{2^n} \right)_{n \in \N}$. La limite de cette suite n'est pas dans $X$.
\par  $\R$ n'est pas séquentiellement compact, car la suite $(n)_{n \in \N}$ ne possède pas de sous-suite convergente.

\begin{theoreme}
  Un espace métrique est compact si et seulement s'il est séquentiellement compact.

  \tcblower
  \begin{preuve}
    Commençons par l'implication vers la droite. Supposons $(X, d)$ un espace métrique compact et $(x_n)_{n \in \N}$ une suite dans $X$. Supposons par l'absurde que $(X, d)$ n'est pas séquentiellement compact. Donc, $(x_n)_{n \in \N}$ n'admet pas de sous-suite convergente (cette suite est donc elle-même pas convergente).
    \par Alors, $\forall x \in X, \exists \delta(x) > 0$ et $N(x) \in \N$ tel que si $n \geq N(x)$, on a que
    $$x_n \not \in B(x, \delta(x)).$$
    Remarquons que $x \in B(x, \delta(x))$ donc
    $$X = \bigcup_{x \in X} \left\{ x \right\} \subseteq \bigcup_{x \in X} B(x, \delta(x)).$$
    Cette dernière union est un recouvrement ouvert de $X$, et comme  $X$ est compact par hypothèse, il existe $x(1), \dots, x(n)$ tel que
    $$X \subseteq \bigcup_{i=1}^n B(x(i), \delta\left( x(i) \right).$$
    Si $n \geq \max\limits_{1 \leq i \leq n} N(i)$, alors on a que
    $$x_n \not \in B\left( x(i), N(i) \right) \quad \forall i \in \{1, \dots, n \}.$$
    Donc, comme $x_n \in X$ et 
    $$x_n \not \in \bigcup_{i=1}^n B(x(i), \delta(x(i)))$$
    C'est une contradiction, car l'union ci-dessus est un recouvrement de $X$ par construction.
  \end{preuve}
\end{theoreme}
Pour la réciproque de ce théorème, nous allons passer par un lemme intermédiaire.

\begin{lemme}
  Supposons que $(X, d)$ est un espace métrique séquentiellement compact et que la collection $\left\{ U_\alpha \right\}_{\alpha \in A}$ est un recouvrement ouvert de $X$. Alors, il existe un $\delta > 0$ tel que, pour chaque  $x \in X$ donnée, il existe $\alpha(x) \in A$ tel que $B(x, \delta) \subseteq U_{\alpha(x)}$.

  \tcblower
  \begin{preuve}
    Supposons par l'absurde que l'affirmation est fausse. Alors, $\forall n \in \N, \exists x_n \in X$ tel que $\forall \alpha(x) \in A, B(x_n, 2^{-n}) \not \subseteq U_{\alpha(x)}$. Par hypothèse, $X$ est séquentiellement compact donc $(x_n)_{n \in \N}$ a une sous-suite convergente dans $X$. Ceci équivaut à écrire que  $\exists x_* \in X$ et $(n_l)_{l \in \N}$ strictement croissant telle que $$\lim_{l \ra \infty} d(x_{n_l}, x_*) = 0.$$
    Comme $x_* \in X$, par hypothèse, il existe $\alpha(x_*) \in A$ tel que $x_* \subseteq U_{\alpha(x_*)}$ car $\left\{ U_\alpha \right\}_{\alpha \in A}$ est un recouvrement de $X$. Comme on est dans un espace métrique, $\exists \delta_* >0$ tel que $B(x_*, \delta_*) \subseteq U_{\alpha(x_*)}$.
    \par Puisque $\left( x_{n_l} \right)_{l \in \N}$ converge vers $x_*$, nous pouvons prendre $l \in \N $ tel que $d(x_{n_l}, x_*) < \frac{\delta_*}{2}$ et tel que $2^{-n_l} \leq \frac{\delta_*}{2}$. Nous avons que $B(x_{n_l}, 2^{-n_l}) \subseteq B(x_*, \delta_*)$ parce que $\forall x\in B(x_{n_l}, 2^{-n_l})$, 
    \begin{align*}
      d(x, x_*) &\leq d(x, x_{n_l}) + d(x_{n_l}, x_*) \\
                &\leq  2^{-n_l} + \frac{\delta_*}{2} \\
                &\leq  \frac{\delta_*}{2} + \frac{\delta_*}{2} = \delta_*.
    \end{align*}
    On a que $B(x_{n_l}, 2^{-n_l}) \subseteq B(x_*, \delta_*) \subseteq U_{\alpha(x_*)}$. Cela mène à une contradiction et achevé la preuve.
  \end{preuve}
\end{lemme}
Reprenons la preuve du théorème précédent.

\begin{propo}
  Si l'espace métrique $(X, d)$ est séquentiellement compact, alors il est compact.

  \tcblower
  \begin{preuve}
    Soit $\left\{ U_\alpha \right\}_{\alpha \in A}$ un recouvrement quelconque de $X$. Par le lemme ci-dessus, il existe $\delta > 0$ tel que $\forall x \in X, \exists \alpha(x) \in A$ tel que $B(x, \delta) \subseteq U_{\alpha(x)}$ car nous supposons $(X, d)$ séquentiellement compact.
    \par Montrons qu'il existe un ensemble $S \subset X$ tel que
    $$X = \bigcup_{s \in S} B(s, \delta).$$
    Supposons par l'absurde que cela est faux: soit $x_0\in X$. Choisissons $x_1$ tel que $x_1 \in X \setminus B(x_0, \delta)$. De même, $x_{n+1} \in X \setminus \bigcup\limits_{i=0}^n B(x_i, \delta)$. Ceci constitue une suite $(x_n)_{n \in \N}$, et par hypothèse de compacité séquentielle, il existe une sous-suite convergente de $(x_n)_{n \in \N}$. Or, ceci est une contradiction, car $d(x_j, x_k) > \varepsilon \forall j \not k$.
    \par Ainsi, il existe $S \subset X$ fini tel que
    $$X = \bigcup_{s \in S} B(s, \delta).$$
    Puisque $\forall s \in S, \exists \alpha(s) \in A$ tel que $B(s, \delta) \subseteq U_{\alpha(s)}$, on a que
    $$X \subseteq \bigcup_{s \in S}B(s, \delta) \subseteq \bigcup_{s \in S} U_{\alpha(s)}.$$
    Et comme $S$ est un ensemble fini, on a bien que $\left\{ U_{\alpha(s)} \right\}_{\alpha(s) \in A}$ est un sous-recouvrement fini. $(X, d)$ est donc compact.
  \end{preuve}
\end{propo}

\section{Connexité}
\subsection{Espaces connexes}

\begin{definition}
  Un espace topologique $(Y, \sigma)$ est dit \textbf{disconnexe} si nous pouvons trouver des ensembles ouverts non-vides $U$ et $V$ tels que $U \cup V = Y$ et $U \cap V = \varnothing$. Un espace qui n'est pas disconnexe est \textbf{connexe}.
\end{definition}

\begin{definition}
  Si $E$ est un sous-ensemble d'un espace topologique $(X, \tau)$, alors $E$ est dit connexe (resp. disconnexe) si la topologie de sous-espace sur $E$ est connexe (resp. disconnexe).
\end{definition}

\begin{lemme}
 Si $E$ est un sous-ensemble d'un espace topologique $(X, \tau)$, alors $E$ est disconnexe si et seulement si nous pouvons trouver des ensembles ouverts  $U$ et $V$ tels que $U \cup V \supseteq E$, $U \cap V \cap E = \varnothing , U \cap E \not = \varnothing$ et $V \cap E \not = \varnothing$.
\end{lemme}
Par exemple, $(X, \tau_\mathrm{ind})$ est clairement connexe, car si $U, V \in \tau_\mathrm{ind}$ avec $U \cup V = X$ et $U \cap V = \varnothing $ alors forcement un des deux ensembles est le vide.

\begin{theoreme}
  Soit $(X, \tau)$ un espace topologique. S'il est connexe et $f: X \ra Y$ est une fonction continue, alors $f(X)$ est connexe.

  \tcblower
  \begin{preuve}
    Démontrons la contraposée: si $f: X \ra Y$ est continue et $f(X)$ disconnexe, alors $X$ est disconnexe.
    \par Par définition de disconnexité, $\exists U, V \in \tau$ tel que $f(X) \subseteq U \cup V$, $U \cap V \cap f(X) = \varnothing$, $U \cap f(X) \not = \varnothing $ et $V \cap f(X) \not = \varnothing $.
    \par On a que $f^{-1}(U) \cup f^{-1}(V) = f^{-1}(U \cup V) \supseteq f^{-1}\left( f(X) \right) \supseteq X$ donc $X \subseteq f^{-1}(U) \cup f^{-1}(V)$.
    \par Par continuité de $f$, on sait que  $f^{-1}(U), f^{-1}(V) \in \tau$. Comme $U \cap f(X) \not = \varnothing $, $\exists x \in X$ tel que $f(x) \in U$ ce qui implique $X \cap f^{-1}(U) \not = \varnothing $. De manière analogue, $X \cap f^{-1}(V) \not = \varnothing$.
    \par Si $x \in f^{-1}(U) \cap f^{-1}(V) \cap f^{-1}\left( f^{-1}(X) \right)$, alors on trouve que
    \begin{align*}
      f(x) &\in f \left( f^{-1}(U) \cap f^{-1}(V) \cap f^{-1}(f(X)) \right) \\
           &\subseteq f(f^{-1}(U)) \cap f(f^{-1}(V)) \cap f(f^{-1}(f(X))) \\
           &\subseteq U \cap V \cap f(X).
    \end{align*}
    Or, ceci est impossible par hypothèse, car $U \cap V \cap f(X) = \varnothing$. Donc, $f^{-1}(U) \cap f^{-1}(V) \cap X = \varnothing $ ce qu'implique que $X$ est disconnexe.
  \end{preuve}
\end{theoreme}

\begin{lemme}\label{lem:loc}
  Soit $(X, \tau)$ un espace topologique et $A$ un ensemble. Soit $\Delta$ la topologie discrete sur $A$. Soit $f: X \ra A$ une fonction. Les affirmations suivantes sont equivalentes.
  \begin{enumerate}
    \item Si $x \in X$ alors nous pouvons trouver un $U \in \tau$ avec $x \in U$ tel que $f$ est constante sur $U$.
    \item Si $x \in A$, alors $f^{-1}(\{ x \}) \in \tau$.
    \item L'application $f: (X, \tau) \ra (A, \Delta)$ est continue.
  \end{enumerate}

  \tcblower
  \begin{preuve}
  Prouvons que l'affirmation 1 implique la 2. Soit $y \in A$ et posons $U = f^{-1}\left( \{ y\} \right)$. Alors par hypothèse, $\forall x \in U, \exists U_x \in \tau$ tel que $x \in U_x$ et $f {\restriction_{U_x}}$ est constante. Comme $\forall x \in U, U_x \subseteq U$, on a que
  $$\bigcup_{x \in U} U_x \subseteq U.$$
  De plus,
  $$U = \bigcup_{x \in U} \left\{ x \right\} \subseteq \bigcup_{x \in U} U_x.$$ 
  Donc $U = \bigcup\limits_{x \in U} U_x$, et comme tout $U_x \in \tau$, il suit que $U \in \tau$.
  \end{preuve}
\end{lemme}

\begin{definition}
  Si les conditions du lemme \ref{lem:loc} sont respectées, nous disons que $f$ est localement constante.
\end{definition}

\begin{theoreme}
  Si $A$ contient au moins deux points, alors un espace topologique $(X, \tau)$ est connexe si et seulement si toute fonction localement constante $f: X \ra A$ est constante.

  \tcblower
  \begin{preuve}
    Prouvons d'abord l'implication vers la droite. Supposons $(X, \tau)$ connexe et $f: (X, \tau) \ra (A, \Delta)$ continue. Comme $\Delta$ est la topologie indiscrète, pour $t \in X$, on a que $\left\{ f(t) \right\}, A \setminus \left\{ f(t) \right\} \in \Delta$ et donc
    \begin{align*}
      U &= \left\{ x \in X \mid f(x) = f(t) \right\} = f^{-1}(\{ f(t)\}) \\
      V &= \left\{ x \in X \mid f(x) \not = f(t) \right\} = f^{-1}( A \setminus \{ f(t)\}) 
    \end{align*}
    sont ouverts. Comme $U \cap V = \varnothing , U \cup V = X$ et $X$ connexe, on a que $V = \varnothing $ et $U = X$ et donc  $f$ est constante par le lemme précédent.

    \par Ensuite, montrons l'implication vers la gauche. Prouvons la contraposée: si $X$ est disconnexe, alors $f: X \ra A$ localement constante n'est pas constante.
    \par Comme $(X, \tau)$ est disconnexe, $\exists U, V \in \tau$ tel que $U \cap V = \varnothing , U \cup V = X$ et $U, V \not = \varnothing $. Choisissons $a, b \in A$, $a \not = b$ et on pose
    $$f(x) = \begin{cases}
      a & \text{ si } x\in U\\
      b & \text{ si } x \in V
    \end{cases}$$
    Nous obtenons alors une fonction continue, localement constante mais pas constante.
  \end{preuve}
\end{theoreme}
\newpage

\begin{figure}[ht]
  \centering
  \incfig{drawing}
  \caption{Chemin $\gamma$ reliant $x$ à $y$}
\end{figure}

\end{document}
