\usepackage[T1]{fontenc}
\usepackage[utf8]{inputenc}
\usepackage{lmodern}
\usepackage[a4paper,margin=2cm]{geometry}
\usepackage{babel}

\usepackage{amsmath, amsfonts, amssymb, mathrsfs, mathtools}
\usepackage{esvect}
\usepackage{systeme}
\usepackage{braket}

\usepackage{fontspec}

% \usepackage{fouriernc}

\usepackage{helvet}
\usepackage{xcolor}
\usepackage{titlesec}
\usepackage{titling}
\usepackage{graphicx}
\usepackage{hyperref}

\usepackage{caption}  
\usepackage{subcaption}

\hypersetup{
	colorlinks=true,
	linkcolor=blue,
	filecolor=magenta,      
	urlcolor=cyan,
	pdfpagemode=FullScreen,
}
\setlength{\fboxrule}{1pt}

\usepackage{tikz, tikz-cd, pgfplots}
\usetikzlibrary{positioning}
\usetikzlibrary{calc}

\newfontfamily\headingfont[]{Helvetica Bold}
% \setmainfont{Helvetica Light}

\usepackage{import}
\usepackage{xifthen}
\usepackage{pdfpages}
\usepackage{transparent}

\newcommand{\incfig}[1]{%
    \def\svgwidth{0.8\columnwidth}
    \import{./figures}{#1.pdf_tex}
}

\titleformat{\chapter}
	{\headingfont\Huge\bfseries}
	{\thechapter}{1em}{\Huge}

\titleformat{\section}
	{\headingfont\LARGE\bfseries}
	{\thesection}{1em}{}
	
\titleformat{\subsection}
	{\headingfont\large\bfseries}
	{\thesubsection}{1em}{}
	

\pretitle{\begin{center}\LARGE\bfseries}
	\posttitle{\par\end{center}\vskip 0.5em}
\preauthor{\begin{center}\large}
	\postauthor{\par\end{center}}
\predate{\begin{center}\large}
	\postdate{\par\end{center}\vskip 0.5em}
	
\newcommand{\R}{\ensuremath{\mathbb{R}}}
\newcommand{\Z}{\ensuremath{\mathbb{Z}}}
\newcommand{\U}{\ensuremath{\mathcal{U}}}
\newcommand{\N}{\ensuremath{\mathbb{N}}}
% \newcommand{\H}{\ensuremath{\mathcal{H}}}
\newcommand{\ra}{\ensuremath{\rightarrow}}
\newcommand{\lra}{\ensuremath{\longrightarrow}}


\renewcommand{\b}[1]{\ensuremath{\mathbf{#1}}}
\renewcommand{\H}{\ensuremath{\mathcal{H}}}

\newcommand{\ps}[2]{\ensuremath{\prescript{}{#1}{#2}}}


\DeclareMathOperator{\rang}{rang\hspace{0.8pt}}
\DeclareMathOperator{\id}{Id}
\DeclareMathOperator{\dist}{dist}
\DeclareMathOperator{\im}{Im}

\newcommand{\class}[1]{\ensuremath{\mathscr{C}^{#1}}}

\DeclareMathOperator{\lenop}{\mathbf{L}}
\newcommand{\len}[2]{\ensuremath{\lenop_{#1}^{#2}}}

\DeclarePairedDelimiter{\abs}{\lvert}{\rvert}%
\DeclarePairedDelimiter{\norm}{\lVert}{\rVert}%

\newcommand{\deriv}[1]{\ensuremath{\frac{\mathrm{d}}{\mathrm{d}#1}}}
\newcommand{\defeq}{\stackrel{\text{def}}{=}}

\renewcommand{\leq}{\leqslant}
\renewcommand{\geq}{\geqslant}
% \renewcommand{\epsilon}{\varepsilon}
\newcommand{\Tau}{\mathcal{T}}

\newcommand{\Int}{\mathrm{Int}}
\newcommand{\Cl}{\mathrm{Cl}}


\title{\headingfont Topologie}	
\author{Karol Gromada}
\date{2024-2025}
